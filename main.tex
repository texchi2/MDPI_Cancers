%  LaTeX support: latex@mdpi.com 
%  For support, please attach all files needed for compiling as well as the log file, and specify your operating system, LaTeX version, and LaTeX editor.

%=================================================================
% Journal of Personalized Medicine, JPM
\documentclass[jpm,article,submit,moreauthors,pdftex]{Definitions/mdpi}
\maxdeadcycles=500 % Output loop---200 consecutive dead cycles.

% For posting an early version of this manuscript as a preprint, you may use "preprints" as the journal and change "submit" to "accept". The document class line would be, e.g., \documentclass[preprints,article,accept,moreauthors,pdftex]{mdpi}. This is especially recommended for submission to arXiv, where line numbers should be removed before posting. For preprints.org, the editorial staff will make this change immediately prior to posting.

%--------------------
% Class Options:
%--------------------
%----------
% journal
%----------
% Choose between the following MDPI journals:
% acoustics, actuators, addictions, admsci, adolescents, aerospace, agriculture, agriengineering, agronomy, ai, algorithms, allergies, analytica, animals, antibiotics, antibodies, antioxidants, appliedchem, applmech, applmicrobiol, applnano, applsci, arts, asi, atmosphere, atoms, audiolres, automation, axioms, batteries, bdcc, behavsci, beverages, biochem, bioengineering, biologics, biology, biomechanics, biomedicines, biomedinformatics, biomimetics, biomolecules, biophysica, biosensors, biotech, birds, bloods, brainsci, buildings, businesses, cancers, carbon, cardiogenetics, catalysts, cells, ceramics, challenges, chemengineering, chemistry, chemosensors, chemproc, children, civileng, cleantechnol, climate, clinpract, clockssleep, cmd, coatings, colloids, compounds, computation, computers, condensedmatter, conservation, constrmater, cosmetics, crops, cryptography, crystals, curroncol, cyber, dairy, data, dentistry, dermato, dermatopathology, designs, diabetology, diagnostics, digital, disabilities, diseases, diversity, dna, drones, dynamics, earth, ebj, ecologies, econometrics, economies, education, ejihpe, electricity, electrochem, electronicmat, electronics, encyclopedia, endocrines, energies, eng, engproc, entropy, environments, environsciproc, epidemiologia, epigenomes, fermentation, fibers, fire, fishes, fluids, foods, forecasting, forensicsci, forests, fractalfract, fuels, futureinternet, futuretransp, futurepharmacol, futurephys, galaxies, games, gases, gastroent, gastrointestdisord, gels, genealogy, genes, geographies, geohazards, geomatics, geosciences, geotechnics, geriatrics, hazardousmatters, healthcare, hearts, hemato, heritage, highthroughput, histories, horticulturae, humanities, hydrogen, hydrology, hygiene, idr, ijerph, ijfs, ijgi, ijms, ijns, ijtm, ijtpp, immuno, informatics, information, infrastructures, inorganics, insects, instruments, inventions, iot, j, jcdd, jcm, jcp, jcs, jdb, jfb, jfmk, jimaging, jintelligence, jlpea, jmmp, jmp, jmse, jne, jnt, jof, joitmc, jor, journalmedia, jox, jpm, jrfm, jsan, jtaer, jzbg, kidney, land, languages, laws, life, liquids, literature, livers, logistics, lubricants, machines, macromol, magnetism, magnetochemistry, make, marinedrugs, materials, materproc, mathematics, mca, measurements, medicina, medicines, medsci, membranes, metabolites, metals, metrology, micro, microarrays, microbiolres, micromachines, microorganisms, minerals, mining, modelling, molbank, molecules, mps, mti, nanoenergyadv, nanomanufacturing, nanomaterials, ncrna, network, neuroglia, neurolint, neurosci, nitrogen, notspecified, nri, nursrep, nutrients, obesities, oceans, ohbm, onco, oncopathology, optics, oral, organics, osteology, oxygen, parasites, parasitologia, particles, pathogens, pathophysiology, pediatrrep, pharmaceuticals, pharmaceutics, pharmacy, philosophies, photochem, photonics, physchem, physics, physiolsci, plants, plasma, pollutants, polymers, polysaccharides, proceedings, processes, prosthesis, proteomes, psych, psychiatryint, publications, quantumrep, quaternary, qubs, radiation, reactions, recycling, regeneration, religions, remotesensing, reports, reprodmed, resources, risks, robotics, safety, sci, scipharm, sensors, separations, sexes, signals, sinusitis, smartcities, sna, societies, socsci, soilsystems, solids, sports, standards, stats, stresses, surfaces, surgeries, suschem, sustainability, symmetry, systems, taxonomy, technologies, telecom, textiles, thermo, tourismhosp, toxics, toxins, transplantology, traumas, tropicalmed, universe, urbansci, uro, vaccines, vehicles, vetsci, vibration, viruses, vision, water, wevj, women, world 

%---------
% article
%---------
% The default type of manuscript is "article", but can be replaced by: 
% abstract, addendum, article, book, bookreview, briefreport, casereport, comment, commentary, communication, conferenceproceedings, correction, conferencereport, entry, expressionofconcern, extendedabstract, datadescriptor, editorial, essay, erratum, hypothesis, interestingimage, obituary, opinion, projectreport, reply, retraction, review, perspective, protocol, shortnote, studyprotocol, systematicreview, supfile, technicalnote, viewpoint, guidelines, registeredreport, tutorial
% supfile = supplementary materials

%----------
% submit
%----------
% The class option "submit" will be changed to "accept" by the Editorial Office when the paper is accepted. This will only make changes to the frontpage (e.g., the logo of the journal will get visible), the headings, and the copyright information. Also, line numbering will be removed. Journal info and pagination for accepted papers will also be assigned by the Editorial Office.

%------------------
% moreauthors
%------------------
% If there is only one author the class option oneauthor should be used. Otherwise use the class option moreauthors.

%---------
% pdftex
%---------
% The option pdftex is for use with pdfLaTeX. If eps figures are used, remove the option pdftex and use LaTeX and dvi2pdf.

%%%%%%%%%%%%%%%%%%%%%%%%%%%%%%%%%% Tex
\usepackage[]{caption} % bf
\newcommand{\bcaption}[2]{\caption{\textbf{#1} #2}}
\usepackage{outlines}
% mark in blue or red
\usepackage{xcolor}

\newenvironment{MyColorPar}[1]{%
    \leavevmode\color{#1}\ignorespaces%
}{%
}%
%\usepackage[upplemac]{inputenc} %applemac support if unicode package fails
%\usepackage[latin1]{inputenc} %UNIX support if unicode package fails
%...
%\usepackage{graphicx}
%\DeclareGraphicsExtensions{.pdf,.png} % for high-resolution PDF image
%\usepackage{hyperref}
\usepackage{xr}
% for xr in your preamble
% https://www.overleaf.com/learn/how-to/Cross_referencing_with_the_xr_package_in_Overleaf
\makeatletter
\newcommand*{\addFileDependency}[1]{% argument=file name and extension
  \typeout{(#1)}
  \@addtofilelist{#1}
  \IfFileExists{#1}{}{\typeout{No file #1.}}
}
\makeatother

\newcommand*{\myexternaldocument}[1]{%
    \externaldocument{#1}%
    \addFileDependency{#1.tex}%
    \addFileDependency{#1.aux}%
}
% my external document I would like to reference the labels of. 
\myexternaldocument{Supplementary_figures} %.tex .aux

%\usepackage{url}

\usepackage{array}
\usepackage{ragged2e}
\usepackage{rotating}
\usepackage{tabularx}
\usepackage{makecell}
% \usepackage{array}
\usepackage{multirow}
\usepackage{colortbl}
\usepackage{hhline}
\usepackage{siunitx} % for  1e-10 scientific notation
%\usepackage{caption}
%\usepackage{subcaption}
\usepackage{booktabs, multirow} % for borders and merged ranges
\usepackage{soul}% for underlines
%\usepackage[table]{xcolor} % for cell colors
\usepackage{changepage,threeparttable} 

%%% for abbreviations, or acronyms
\usepackage[automake, acronym, nopostdot]{glossaries} 
\usepackage{glossary-inline}
%\setacronymstyle{long-short}
%\renewcommand*{\glossarysection}[2][]{} 
%\renewcommand*{\glossarysection}[2][]{\textbf{#1}: }
% for abbreviations environment
%\newcommand{\abbrlabel}[1]{\makebox[3cm][l]{\textbf{#1}\ \dotfill}}
\newenvironment{abbreviation}%{\begin{list}{}{\renewcommand{\makelabel}{\abbrlabel}}}{\end{list}}
% \newenvironment{<name>}[<number>][<default>]


\makeglossaries %https://tex.stackexchange.com/questions/110095/list-of-acronyms-is-not-displayed

\newacronym{ihc}{IHC}{immunohistochemistry}
\newacronym{fdr}{FDR}{false discovery rate}

\newacronym{hpa}{HPA}{the Human Protein Atlas}
\newacronym{hnscc}{HNSCC}{head and neck squamous cell carcinoma}
\newacronym{tcga}{TCGA}{the Cancer Genome Atlas}
\newacronym{tcpa}{TCPA}{the Cancer Proteome Atlas}
\newacronym{rna}{RNA}{ribonucleic acid}
\newacronym{rnaseq}{RNA-Seq}{RNA sequencing}
\newacronym{lncrna}{lncRNA}{long non-coding RNA}
%\newacronym{km}{KM}{Kaplan-Meier}
\newacronym{rppa}{RPPAs}{reverse-phase protein arrays}
\newacronym{rpma}{RPMA}{reverse-phase protein lysate microarray}

\newacronym{mmp}{MMP}{matrix metalloproteinase}
 %DKK1, CAMK2N1, STC2, PGK1, SURF4, USP10, NDFIP1, FOXA2, STIP1, and DKC1
 %ZNF557, ZNF266, IL19, MYO1H, FCGBP, LOC148709, EVPLL, PNMA5, KIAA1683, and NPB

\newacronym{DKK1}{DKK1}{dickkopf WNT signaling pathway inhibitor 1} 
\newacronym{CAMK2N1}{CAMK2N1}{calcium/calmodulin dependent protein kinase II inhibitor 1} 
\newacronym{STC2}{STC2}{stanniocalcin 2} 
\newacronym{PGK1}{PGK1}{phosphoglycerate kinase 1} 
\newacronym{SURF4}{SURF4}{surfeit 4} 
\newacronym{USP10}{USP10}{ubiquitin specific peptidase 10} 
\newacronym{NEDD4}{NEDD4}{neural precursor cell expressed, developmentally down-regulated 4}
\newacronym{NDFIP1}{NDFIP1}{NEDD4 family interacting protein 1} 
\newacronym{FOXA2}{FOXA2}{forkhead box A2} 
\newacronym{STIP1}{STIP1}{stress-induced-phosphoprotein 1} 
\newacronym{DKC1}{DKC1}{dyskeratosis congenita 1, dyskerin} 

\newacronym{ZNF557}{ZNF557}{zinc finger protein 557} 
\newacronym{ZNF266}{ZNF266}{zinc finger protein 266} 
\newacronym{IL19}{IL19}{interleukin 19} 
\newacronym{MYO1H}{MYO1H}{myosin 1H} 
\newacronym{FCGBP}{FCGBP}{Fc fragment of IgG binding protein} 
\newacronym{LOC148709}{LOC148709}{LncRNA LOC148709} 
\newacronym{EVPLL}{EVPLL}{envoplakin-like protein} 
\newacronym{PNMA5}{PNMA5}{paraneoplastic antigen like 5} 
%\newacronym{KIAA1683}{KIAA1683}{IQCN, IQ Motif Containing N} 
\newacronym{IQCN}{IQCN}{IQ motif containing N} % previous name KIAA1683
% "IQ" refers to the first two amino acids of the motif: isoleucine (commonly) and glutamine (invariably)
\newacronym{NPB}{NPB}{neuropeptide B} 

 \newacronym{rt}{RT}{radiation therapy}
 \newacronym{nccn}{NCCN}{National Comprehensive Cancer Network}
 \newacronym{hif}{HIF}{hypoxia-inducible factor}
 \newacronym{egfr}{EGFR}{epidermal growth factor receptor}
 \newacronym{ras}{RAS}{rat sarcoma}
 \newacronym{hras}{HRAS}{Harvey rat sarcoma viral oncoprotein}
 \newacronym{erk}{ERK}{extracellular signal-regulated kinases}
 \newacronym{us}{US}{United States}
 \newacronym{fda}{FDA}{Food and Drug Administration}
 \newacronym{tpf}{Tax-PF}{docetaxel, cisplatin, and 5-fluorouracil}
 \newacronym{tki}{TKI}{tyrosine kinase inhibitor}
 \newacronym{her}{HER}{human epidermal growth factor receptor}
 \newacronym{ici}{ICI}{immune-checkpoint inhibitor}
 \newacronym{ctla4}{CTLA-4}{cytotoxic T lymphocyte antigen 4}
 \newacronym{pd1}{PD-1}{programmed death 1}
 \newacronym{pdl1}{PD-L1}{programmed death ligand 1}
 \newacronym{tim3}{TIM-3}{T-cell immunoglobulin mucin protein 3}
 \newacronym{lag3}{LAG-3}{lymphocyte activation gene 3}
 \newacronym{ifng}{IFN-$\gamma$}{interferon gamma}
 \newacronym{tigit}{TIGIT}{T cell immunoglobin and immunoreceptor tyrosine-based inhibitory motif}
 \newacronym{gitr}{GITR}{glucocorticoid-induced tumor necrosis factor receptor}
 \newacronym{vista}{VISTA}{V-domain Ig suppressor of T-cell activation}
 \newacronym{tmsb4x}{TMSB4X}{thymosin beta-4 X-linked}
 \newacronym{emt}{EMT}{epithelial-mesenchymal-transition}
 \newacronym{gdc}{GDC}{Genomic Data Commons}
 \newacronym{nci}{NCI}{the National Cancer Institute}
 \newacronym{gdac}{GDAC}{Genome Data Analysis Center}
 \newacronym{rest}{REST}{Representational State Transfer} 
 \newacronym{api}{API}{Application Programmable Interface}
\newacronym{grch38}{GRCh38}{Genome Reference Consortium Homo sapiens genome assembly 38}
\newacronym{fpkm}{FPKM}{Fragments per kilobase per million reads mapped}
\newacronym{rsem}{RSEM}{RNA-Seq by Expectation-Maximization}
\newacronym{slca}{SLC35E2A}{solute carrier family 35 member E2A}
\newacronym{slcb}{SLC35E2B}{solute carrier family 35 member E2B}
\newacronym{cde}{CDE}{Common Data Element}
\newacronym{id}{ID}{identification}
\newacronym{ajcc}{AJCC}{the American Joint Committee on Cancer}
\newacronym{uicc}{UICC}{he Union for International Cancer Control}
\newacronym{tnm}{TNM}{the tumor size (T), cervical lymph node metastases (N), and distal metastasis status (M)}
\newacronym{ci95}{95\% CI}{95\% confidence interval}
\newacronym{os}{OS}{overall survival}
\newacronym{hr}{HR}{hazard ratio}
\newacronym{hpv}{HPV}{human papillomavirus}
\newacronym{ene}{ENE}{extra-nodal extension}
\newacronym{lvsi}{LVSI}{lymph-vascular space invasion}
\newacronym{pni}{PNI}{perineural invasion}
\newacronym{doi}{DOI}{depth of invasion}
\newacronym{lnd}{LND}{lymph node density}
\newacronym{wpoi5}{WPOI-5}{worst pattern of invasion score 5}
\newacronym{glut4}{GLUT4}{glucose transporters 4}
\newacronym{slc2a4}{SLC2A4}{solute carrier family 2 member A4}
\newacronym{trim24}{TRIM24}{tripartite motif-containing 24}
\newacronym{til}{TIL}{tumor-infiltrating lymphocytes}
\newacronym{tmb}{TMB}{tumor mutational burden}


%=================================================================
\firstpage{1} 
\makeatletter 
\setcounter{page}{\@firstpage} 
\makeatother
\pubvolume{1}
\issuenum{1}
\articlenumber{0}
\pubyear{2021}
\copyrightyear{2021}
%\externaleditor{Academic Editor: Firstname Lastname}
\datereceived{} 
\dateaccepted{} 
\datepublished{} 

%------------------------------------------------------------------
% The following line should be uncommented if the LaTeX file is uploaded to arXiv.org
%\pdfoutput=1

%=================================================================
% Add packages and commands here. The following packages are loaded in our class file: fontenc, inputenc, calc, indentfirst, fancyhdr, graphicx, epstopdf, lastpage, ifthen, lineno, float, amsmath, setspace, enumitem, mathpazo, booktabs, titlesec, etoolbox, tabto, xcolor, soul, multirow, microtype, tikz, totcount, changepage, paracol, attrib, upgreek, cleveref, amsthm, hyphenat, natbib, hyperref, footmisc, url, geometry, newfloat, caption

%=================================================================
%% Please use the following mathematics environments: Theorem, Lemma, Corollary, Proposition, Characterization, Property, Problem, Example, ExamplesandDefinitions, Hypothesis, Remark, Definition, Notation, Assumption
%% For proofs, please use the proof environment (the amsthm package is loaded by the MDPI class).

%=================================================================
% Full title of the paper (Capitalized)
\Title{A Global Genome-wide Scan with Optimal Cutoff Mining for Emerging Biomarkers in Head and Neck Squamous Cell Carcinoma}

% MDPI internal command: Title for citation in the left column
\TitleCitation{A Global Genome-wide Scan with Optimal Cutoff Mining for Emerging Biomarkers in Head and Neck Squamous Cell Carcinoma}

% Author Orchid ID: enter ID or remove command
\newcommand{\orcidauthorA}{0000-0002-4476-2600} % Add \orcidA{} behind the author's name
\newcommand{\orcidauthorB}{0000-0001-6497-4232} % Add \orcidB{} behind the author's name

% Authors, for the paper (add full first names)
\Author{
Li-Hsing Chi $^{1,2}$\orcidA{}, Alexander TH Wu $^{1}$,
Michael Hsiao $^{3}$*
and Yu-Chuan (Jack) Li $^{1,4}$*\orcidB{}}
%\Author{Firstname Lastname $^{1,\dagger,\ddagger}$\orcidA{}, Firstname Lastname $^{1,\ddagger}$ and Firstname Lastname $^{2,}$*}

% MDPI internal command: Authors, for metadata in PDF
\AuthorNames{Li-Hsing Chi, Alexander TH Wu, Michael Hsiao and Yu-Chuan (Jack) Li}

% MDPI internal command: Authors, for citation in the left column
\AuthorCitation{Chi, LH.; Wu, ATH.; Hsiao, M.; Li, YCJ.}

% Affiliations / Addresses (Add [1] after \address if there is only one affiliation.)
\address{%
$^{1}$ \quad The Ph.D. Program for Translational Medicine, College of Medical Science and Technology\unskip, 
    Taipei Medical University and Academia Sinica\unskip, Taipei\unskip, Taiwan\\
$^{2}$ \quad Division of Oral and Maxillofacial Surgery, Department of Dentistry\unskip,
    Taipei Medical University Hospital\unskip, Taipei\unskip, Taiwan\\
$^{3}$ \quad Genomics Research Center\unskip, 
    Academia Sinica\unskip, Taipei\unskip, Taiwan\\
$^{4}$ \quad Graduate Institute of Biomedical Informatics, College of Medical Science and Technology\unskip, Taipei Medical University\unskip, Taipei\unskip, Taiwan\\
}

% Contact information of the corresponding author
\corres{Correspondence: Hsiao: mhsiao@gate.sinica.edu.tw; Li: jaak88@gmail.com}
%(optional; include country code; if there are multiple corresponding authors, add author initials) +xx-xxxx-xxx-xxxx (F.L.)}

% Current address and/or shared authorship
%\firstnote{Current address: Affiliation 3} 
%\secondnote{These authors contributed equally to this work.}
% The commands \thirdnote{} till \eighthnote{} are available for further notes

%\simplesumm{} % Simple summary

%\conference{} % An extended version of a conference paper

% Abstract (Do not insert blank lines, i.e. \\) 
\abstract{
Survival analysis using the Cancer Genome Atlas (TCGA) dataset is a well-known method to discover gene expression-based prognostic biomarkers of head and neck squamous cell carcinoma (HNSCC). It is necessary to determine a cutoff point by patients' dichotomization for the continuous gene expression. Usually, an optimization algorithm for cutoff determination has been set at the median or quantiles of RNA sequencing value.
%There are few clinicopathological features available on those pre-processed datasets. (feature selection or feature engineering issue)
We developed a comprehensive workflow to perform biomarker discovery. It includes data retrieval, data pre-processing, feature selection, cutoff engine, Kaplan-Meier survival analysis, and Cox proportional hazard modeling.
Using this workflow on the TCGA HNSCC cohort, we scanned human protein-coding genes and adjusted them with confounders, and Bonferroni corrected \textit{P} value. After validation with an independent \acrshort{hnscc} cohort (GSE2837), the result showed six overexpressed genes (symbol as CAMK2N1, PGK1, SURF4, USP10, NDFIP1, FOXA2) are significantly associated with a poor prognosis of overall survival. 
Furthermore, the four overexpressed genes (symbol as IL19, FCGBP, IQCN - former symbol as KIAA1683, and NPB) are correlated with better survival.
We suggest this workflow equipped with a cutoff-mining engine could help for biomarker discovery.
}
% validated by the other HNSCC cohort:
%1) poor prognosis: CAMK2N1 (0.048214), PGK1 (0.009978), SURF4 (0.023127), USP10 (0.017768), NDFIP1 (0.022758), FOXA2 (0.001587);\\ % FOXA2 (0.038125)
%2) better prognosis: IL19 (0.049731), FCGBP (0.005658), KIAA1683 (IQCN, 0.005886), NPB (0.014177);\\

% row 20: The result showed ten overexpressed genes (symbol as DKK1, CAMK2N1, STC2, PGK1, SURF4, USP10, NDFIP1, FOXA2, STIP1, and DKC1) are significantly associated with a poor prognosis of overall survival. Furthermore, the other ten overexpressed genes (symbol as ZNF557, ZNF266, IL19, MYO1H, FCGBP, LOC148709, EVPLL, PNMA5, IQCN - former symbol as KIAA1683, and NPB) are correlated with better survival.


% Keywords
\keyword{
Head and Neck Squamous Cell Carcinoma (HNSCC);
%Genome Database\sep
the Cancer Genome Atlas (TCGA);
RNA-sequencing;
Survival Analysis;
Optimal Cutoff;
%Biomarker\sep % Discovery\sep
%Tumor Type-agnostic Therapy\sep
%Immuno-Oncology\sep
%Targeted Therapy\sep
%Systemic Therapy\sep
%Surgical Margin
}

% I have a graphic abstract: graphic_abstract_pvalueTex.pdf
% National Cancer Institute. Changing the dressing on a patient’s neck. https://unsplash.com/photos/IrqLPsxemQ4?utm_source=unsplash&utm_medium=referral&utm_content=creditShareLink (1950).\cite{nci1950}
%%%%%%%%%%%%%%%%%%%%%%%%%%%%%%%%%%%%%%%%%%


\begin{document}
%\begin{paracol}{2}
%%%%%%%%%%%%%%%%%%%%%%%%%%%%%%%%%%%%%%%%%%
%\setcounter{section}{-1} %% Remove this when starting to work on the template.
%\section{How to Use this Template}

%The template details the sections that can be used in a manuscript. Note that the order and names of article sections may differ from the requirements of the journal (e.g., the positioning of the Materials and Methods section). Please check the instructions on the authors' page of the journal to verify the correct order and names. For any questions, please contact the editorial office of the journal or support@mdpi.com. For LaTeX-related questions please contact latex@mdpi.com.
%The order of the section titles is: Introduction, Materials and Methods, Results, Discussion, Conclusions for these journals: aerospace,algorithms,antibodies,antioxidants,atmosphere,axioms,biomedicines,carbon,crystals,designs,diagnostics,environments,fermentation,fluids,forests,fractalfract,informatics,information,inventions,jfmk,jrfm,lubricants,neonatalscreening,neuroglia,particles,pharmaceutics,polymers,processes,technologies,viruses,vision
%%

%\begin{figure}
%    \centering
%    \includegraphics[width=14cm]{graphic_abstract_pvalueTex.pdf}
%    \caption{(inset image courtesy of National Cancer Institute\cite{nci1950})}
%    \label{fig:my_label}
%\end{figure}

\section{Introduction}

Head and neck squamous cell carcinoma (\acrshort{hnscc}), including oral, oropharyngeal, and hypopharyngeal origin, is the fourth leading cancer causes of death for males in Taiwan\cite{MOHW_death2017}. The age-standardized incidence rate of \acrshort{hnscc} in males is 42.43 per 100,000 persons\cite{MOHW_incidence2018}. 
% chemotherapy => systemic therapy
The treatment strategies of \acrshort{hnscc} are surgery alone, systemic therapy with concurrent radiation therapy (systemic therapy/\acrshort{rt}), or surgery with adjuvant systemic therapy/\acrshort{rt} (according to \acrlong{nccn}, \acrshort{nccn} Clinical Practice Guidelines in \acrshort{hnscc}, Version 2.2020)\cite{Pfister2020a}. Despite the improvement in those interventions, the survival of \acrshort{hnscc} has improved only marginally over the past decade worldwide\cite{hpa2019}. The critical advancement of targeted therapy and immuno-oncology should benefit from emerging prognostic  biomarkers guiding modern systemic therapy.

%latex https://www.latex-tutorial.com/symbols/greek-alphabet/
%\subsection*{Linear regression in biomarker discovery}
Linear regression is useful as a concept in biomarker discovery.
A linear regression equation that relates a Y-variable to n X-variables is written as\\[1cm]
%\par
%Highlights
% https://quicklatex.com
\begin{flushleft}
% equation
%\begin{math}

ground truth Y (e.x. patient's survival):\\[0.5cm]
%Y$:\\[0.3cm]
$Y = \beta_0 + \beta_1 X_1 + \beta_2 X_2 + \beta_3 X_3 + ... + \beta_n X_n + \epsilon$
\linebreak
\linebreak
Input $X_1...X_n$ (e.x. patients features: such as age, gender, gene expression, cancer stage, or even spiritual, emotional, and social status)
\\[0.5cm]
Coefficients $\beta_0...\beta_n$ (built in model)\\[0.5cm]
A residual (error) term is calculated as $\epsilon=Y - \hat Y$\\
Since the model established by survival analysis, predicted value $\hat Y$ is calculated as:\\[0.3cm]
$\hat Y = \beta_0 + \beta_1 X_1 + \beta_2 X_2 + \beta_3 X_3 + ... + \beta_n X_n$
\linebreak
%(predicted Y, output) intercept, slope, independent (input) \Times
%\end{math}
\end{flushleft}
Thus, the biomarker discovery strategy is survival modeling through collections of $X_1...X_n$ features from cancer datasets.\\[1cm]



%% TP53 gene => p53 protein; mutation of TP53
Accumulative knowledge showed that some biomarkers have prognostic significance in \acrshort{hnscc}. For example, node-negative \acrshort{hnscc} patients with p53 overexpression were found to hold lower survival\cite{DeVicente2004}.
Overexpression of \acrfull{hif}-1 alpha\cite{Aebersold2001} or Ki-67\cite{Couture2002} was found to be correlated with poor response to radiotherapy of \acrshort{hnscc}. The \acrfull{egfr}\cite{O-Charoenrat2000}\cite{Bentzen2005} and \acrfull{mmp}\cite{Harrington2017} were found to be overexpressed to promote invasion and metastasis of \acrshort{hnscc}.
From 2000 to 2006, the first anti-\acrshort{egfr} antibody-drug (cetuximab) has been developed and combined with radiotherapy, known as bio-\acrshort{rt}, to increase survival of unresectable locoregionally advanced disease\cite{Bonner2006a}.
The systemic therapy of cetuximab plus platinum-fluorouracil chemotherapy (EXTREME regimen) has improved overall survival when given as first-line treatment in patients with recurrent or metastatic \acrshort{hnscc}\cite{Vermorken2008}\cite{Rivera2009}. It has been approved by the \acrshort{us} \acrfull{fda} since 2008. In advance, the bio-\acrshort{rt} could be proceeded with \acrfull{tpf} induction chemotherapy to overcome  radio-resistance of \acrshort{hnscc}\cite{Blanchard2013}.

However, Rampias and his colleagues\cite{Rampias2014} suggested that \acrfull{hras} mutations could mediate cetuximab resistance in systemic therapy of \acrshort{hnscc} via the \acrshort{egfr}/\acrfull{ras}/\acrfull{erk} signaling pathway.
After that, the \acrshort{egfr} \acrfull{tki} was introduced to help cetuximab in 2018. The anti-tumor activity was observed in a phase 1 trial for \acrshort{hnscc} patients using cetuximab and afatinib, a \acrshort{tki} of \acrshort{egfr}, \acrfull{her}2, and \acrshort{her}4\cite{Gazzah2018}. Other \acrshort{egfr} \acrshort{tki}s, such as gefitinib, erlotinib, osimertinib, were also developed to treat advanced \acrshort{hnscc}.
Although 90\% of \acrshort{hnscc} has overexpression of \acrshort{egfr}, cetuximab has only 10\% to 20\% response rate on those patients. So far, cetuximab is still the only drug of choice with proven efficacy, which has targeted the selected \acrshort{hnscc} patients\cite{Taberna2019}.

Until the immuno-oncology era, \acrfull{ici} was introduced since 2014 for treating \acrshort{hnscc}\cite{Seiwert2014}\cite{Swanson2015}.
The \acrshort{ici} works on immune checkpoint molecules, including \acrfull{pd1}, \acrfull{ctla4}, \acrfull{tim3}, \acrfull{lag3}, \acrfull{tigit}, \acrfull{gitr}, and \acrfull{vista}\cite{Mei2020}.
The \acrshort{us} \acrshort{fda} has approved the anti-\acrshort{pd1} agents (e.g. pembrolizumab and nivolumab) as a monotherapy for the platinum-treated patients with recurrent or metastatic \acrshort{hnscc}\cite{Cramer2019}. 
According to the phase 3 KEYNOTE-048 study, \acrshort{pdl1} is a validated biomarker used in clinical guidance for candidate selection of pembrolizumab\cite{Burtness2019}\cite{Gavrielatou2020}.
However, because of the complexity of immune-tumor interaction, \acrshort{ici} has 20\% response rate to \acrfull{pdl1} expressed patients (over 50\% in \acrlong{ihc}, \acrshort{ihc} staining of \acrshort{hnscc})\cite{Swanson2015}\cite{Gavrielatou2020}.

According to our proteomic study between 2010 to 2017, \acrfull{tmsb4x} was reported to be related to tumor growth and metastasis of \acrshort{hnscc}\cite{Chi2017}. It was also reported by the subsequent investigations that \acrshort{tmsb4x} has engaged in tumor aggressiveness through \acrfull{emt} on pancreatic\cite{Zhang2008}, gastric\cite{Ryu2012}, colorectal\cite{Gemoll2015}, lung\cite{Huang2016}, ovarian\cite{Chu2019}, and melanoma\cite{Makowiecka2019} cancers. Thus, it might be suggested that \acrshort{tmsb4x}  is a candidate for tumor type-agnostic therapy\cite{Yan2018} as a common biomarker crossing several types of cancer.
%TMSB4X proteins regulate intracellular signal transduction and has been proved to overexpress in various cancers, including colorectal, lung, gastric, pancreatic, and squamous cell cancers.\cite{Chi2017}
% https://doi.org/10.1038/s41598-017-09539-w


%\subsection{A introduction of TCGA}
%\begin{MyColorPar}{red}
%%%%%% revision1 added [TCGA] [2021/02/02]
A large-scale cancer database, aggregating many $X_1 ... X_n$ features, is necessary to facilitate the biomarker discovery.
The Cancer Genome Atlas (TCGA) project\cite{Weinstein2013} has been developed since 2005 and supervised by the National Cancer Institute's (NCI) Center for Cancer Genomics and the National Human Genome Research Institute (NHGRI), funded by the US government.
TCGA represents comprehensive genomics and clinic data from 84,392 patients among 33 major cancer types (data release 27.0 - Oct 29, 2020, available at https://www.cancer.gov/about-nci/organization/ccg/research/structural-genomics/tcga/studied-cancers).
%The Cancer Genome Atlas (TCGA): The TCGA consortium, which is a National Institute of Health (NIH) initiative, makes publicly available molecular and clinical information for more than 33 types of human cancers including exome (variant analysis), single nucleotide polymorphism (SNP), DNA methylation, transcriptome (mRNA), microRNA (miRNA) and proteome.
TCGA and other collaborated \acrfull{gdac} generated and analyzed DNA (mutation, copy number variation, methylation, simple nucleotide polymorphism, SNP), RNA (microarray, RNA-Seq, microRNA), and protein (reverse protein phased array) data derived from biospecimen. Sample types available at TCGA are primary solid tumors, recurrent solid tumors, blood-derived normal and tumor, metastatic, and solid tissue normal.  

The NCI's Genomic Data Commons (GDC, available at https://portal.gdc.cancer.gov) receives, processes, and distributes genomic, clinical, and biospecimen data from TCGA database and other cancer research programs. The clinical features has been defined by \acrshort{tcga} \acrshort{gdc} data dictionary: \acrfull{cde}\cite{CDE2019}. The \acrshort{rnaseq} expression data has been harmonized and re-aligned against an official reference genome build (\acrlong{grch38}, \acrshort{grch38}).
%The old TCGA data portal (https://tcga-data.nci.nih.gov/docs/publications/tcga/) stops updating and all TCGA data was transferred to the GDC data portal .
\acrshort{tcga}, GDC and some research communities offer several computational tools to the public for facilitating cancer research. %There is three categories of tools to approach the TCGA data from GDC.
GDC Data Portal is the official web-based TCGA data analysis tools. Other available web-based tools have been reviewed by Zhang et al.\cite{Zhang2019b} and 
Matthieu Foll (availalbe at https://github.com/IARCbioinfo/awesome-TCGA).
One of \acrshort{gdac}s, the Broad TCGA Data and Analyses (Broad GDAC), provides Firehose, a repository of the TCGA public-accessible Level 3 data and Level 4 analyses. Broad GDAC Firehose is an analytical infrastructure that analyses algorithms not performed by the GDC (e.x. GISTIC, MutSig2CV, correlation with clinical variables, mRNA clustering). 
A web-based version of Broad GDAC Firehose is Firebrowse (available at firebrowse.org, Version: 1.1.40, 2019-10-13).
%Python and UNIX bindings
%R bindings
Broad GDAC Firebrowse provides graphical tools like viewGene to explore expression levels and iCoMut to explore a mutation analysis of each TCGA disease. 

%Second, GDC data transfer tool (available at https://gdc.cancer.gov/access-data/gdc-data-transfer-tool) allows users to access the open or controlled data. %the GDC data model, data formats
% RESTful API

GDC \acrfull{api} is designed under the \acrfull{rest} architecture and provides accessibility to external users for programmatic access to the same functionality found through GDC Portals. Those functions include searching, viewing, submitting, and downloading subsets of data files, metadata, and annotations based on specific parameters. Moreover, if restricted data is requested, the user must provide a token along with the API call. This token can be downloaded directly from the GDC Portals. (paragraph is provided here courtesy of the National Cancer Institute)
% Articles from Cureus are provided here courtesy of Cureus Inc.
Broad GDAC Firebrowse \acrshort{rest}ful \acrshort{api} could be accessed using an R package, FirebrowseR (available at https://github.com/mariodeng/FirebrowseR)\cite{Deng2017}.

% clinical
%We utilized FirebrowseR's function call, Samples.Clinical(cohort = "HNSC", format="csv"), to get all 81 clinical features (including pathological data, defined by \acrshort{tcga} \acrshort{gdc} data dictionary: \acrfull{cde}\cite{CDE2019}) of all 528 \acrshort{hnscc} patient.

% to summarize the advantages of its use.
The highlights of the advantages of applying the TCGA data for cancer biomarker identification include:
\begin{outline}
\1  To the best of our knowledge, the TCGA database is the largest collection (in both cancer types and cohort size) of comprehensive genomics with survival data available in the field of cancer research so far.  
There are many physical and social features ($X_1 ... X_n$) of patients available for survival modeling.
The whole-genome sequencing data has been harmonized across all GDACs.  Many databases adopt the essential demographic data from TCGA since it has comprehensive clinical features, such as exposure to alcohol, asbestos, radioactive radon, tobacco smoking, or cigarettes.
\1  TCGA has a remarkable advantage for computational and life scientists who study cancer since useful web-based tools and API are ready to analyze and visualize TCGA data. It might be getting help quickly from the research community for trouble-shooting.
\1  Many achievements of diagnoses, treatments, and prevention with the TCGA data have already been published and keep growing\cite{Tomczak2015}.\\[0.5cm]
\end{outline}



In summary, identifying predictive biomarkers for selecting standard-of-care or advanced systemic therapy\cite{Cristina2019} in \acrshort{hnscc} is crucial. 
However, there are three challenges of biomarker discovery from survival analysis.
First, although \acrshort{tcga} genomics data was harmonized, there is unclean data, including null expressed genes, over 50\% of the cohort, should be manually investigated and cleaned.
Second, we need to find a way to determine candidates from the expression level of 20,500 human protein-coding genes\cite{Clamp2007}. Usually, the investigators should get the rationale or revelation of the genes of interest on a specific cancer type. They should upload those genes manually onto bioinformatics tools, such as SurvExpress (http://bioinformatica.mty.itesm.mx:8080/Biomatec/ \newline
SurvivaX.jsp\cite{Aguirre-Gamboa2013}, which has been lost since Oct/2019 and currently out of funds), and analyze the cohort (e.x. \acrshort{tcga}). After downloading the survival results, they could curate plots and tables carefully.
It is not possible to scan the whole human protein-coding genome in this way.
Third, we need to find an optimal cutpoint of that \acrshort{rna} expression data to maximize candidate mining coverage. The above mentioned web-based tools might set a cutpoint at the median, 1/4 quantile, or 3/4 quantile for subsequent analyses. There are several visualization software or R packages which deal with cutoff determination, such as Prognoscan\cite{Mizuno2009a}, Cutoff Finder\cite{Budczies2012}, Findcut\cite{Chang2017a}, Human protein atlas\cite{Uhlen2017}, OptimalCutpoints\cite{Cristina2019},  cutpointr (available at https://github.com/thie1e/ \newline
cutpointr), and cutoffR (available at https://cran.r-project.org/web/
packages/cutoffR). However, non of them could combine the scanning of the protein-coding genes and cutoff optimization programmatically.

Our approach describes a comprehensive workflow implemented in the R script, which runs on the Rstudio server.
Its function includes data retrieval, data pre-processing, feature selection, cutoff mining engine, Kaplan-Meier survival analysis, Cox proportional hazard modeling, and biomarker selection.
The workflow, shown in Figure \ref{fig:figure1}, scanned 20,500 human protein-coding genes of the \acrshort{tcga} \acrshort{hnscc} cohort to yield a model with biomarker estimate.

%We found that the surgical margin status and tobacco exposure are independent risk factors of patient survival. 
%Our findings also suggested several candidate biomarkers are associated with the prognosis of overall survival (OS) under optimal cutoff point with significant P-value. Those genes may be potential therapeutic targets of HNSCC.

%\subsubsection{Figure 1:h}
\begin{figure}[hp]
\centering
\includegraphics[width=14cm]{Figure_1_manuscript_workflow} % .PDF is better than .png
%, height=8cm
%\caption
\bcaption{A workflow of \acrshort{hnscc} biomarker discovery, step 1 (\textcolor{blue}{blue line}: main procedure) and step 2 (\textcolor{orange}{orange line}: analysis export).}
%Step 3 (purple line: dealing with surgical margin).
{The "main procedure" includes data retrieval from TCGA GDC data portal, data process with merging and cleaning (red rectangled),% should be modified aggressively according to different sources of database)
then performing the survival analyses. The Cutoff engine (cutofFinder\_func.HNSCC.R) might calculate all possible Kaplan-Meier \textit{P} value to find the optimal cutoff value of RNA-Seq for subsequent Cox modeling (a draft diagram shown on the upper right corner "HNSCC cohort", the serial cut for grouping patients with \textcolor{green}{low [green]} or \textcolor{red}{high [red]} expression of a specific gene, to yield a collection of \textit{P} values; please see Materials and Methods section for details). The step 2 "analysis export" performs dissecting and selection of candidate genes by Bonferroni adjusted \textit{P} value as well as a hazard ratio of Cox model, based on the results from the step 1.
(HNSCC: head and neck squamous cell carcinoma; TCGA: the Cancer Genome Atlas; RNA-Seq: RNA sequencing; GDC: Genomic Data Commons.)}
\label{fig:figure1}
\end{figure}

\clearpage
%%%%%%%%%%%%%%%%%%%%%%%%%%%%%%%%%%%%%%%%%%

\section{Results}


The 9416 Kaplan-Meier plots with associated Cox's univariate and multivariate tables were generated at workflow step 1 (see Figure \ref{fig:figure1}) and justified by the ranking of hazard ratios.
The 967 genes were kept by criteria of uncorrected \textit{P} value (below 0.05) and \acrfull{hr} (greater than 1.5 or less than 0.5) (see Figure \ref{fig:figure2}(a) univariate, and Figure \ref{fig:figure2}(b) multivariate plots). 
At the final step, a stringent criteria (Bonferroni \textit{P} value correction) was used to yield twenty candidates (see Figure \ref{fig:figure2}(c), (d)). 
% before validation
Among these candidates, ten overexpressed genes (symbol as DKK1, CAMK2N1, STC2, PGK1, SURF4, USP10, NDFIP1, FOXA2, STIP1, and DKC1) are significantly associated with a poor prognosis of  \acrshort{os} (see Table \ref{table:table1}). 
The other ten overexpressed genes (symbol as ZNF557, ZNF266, IL19, MYO1H, FCGBP, LOC148709, EVPLL, PNMA5, IQCN - former symbol as KIAA1683, and NPB) are correlated with better survival (see Table \ref{table:table3}, with their gene name).
We made a volcano plot for 9416 genes by Kaplan-Meier \textit{P} value (less than 0.05, obtained during cutoff finding procedure) against the Cox hazard ratio (see Figure \ref{fig:figure3}). The plot reveals that the most significant (Bonferroni-adjusted \textit{P} $< 0.05$) candidate genes are located above the dotted line. At the same time, Cox's HR separated them on the two-side with prognostic impact.
% => after validation: The result showed six overexpressed genes (symbol as CAMK2N1, PGK1, SURF4, USP10, NDFIP1, FOXA2) are significantly associated with a poor prognosis of overall survival. Furthermore, the four overexpressed genes (symbol as IL19, FCGBP, IQCN - former symbol as KIAA1683, and NPB) are correlated with better survival.

%\subsubsection{Figure 2:h}
\begin{figure}[hp]
\centering
\includegraphics[width=14cm]{Figure2.pdf}
\bcaption{\acrshort{hnscc} Cox's hazard ratio and \textit{P} value plot.}
{(a) Univariate HR versus uncorrected \textit{P} value; (b) Multivariate HR versus uncorrected \textit{P} value; (c) Univariate HR versus Bonferroni corrected \textit{P} value; and (d) Multivariate HR versus Bonferroni corrected \textit{P} value.}
\label{fig:figure2}
\end{figure}

% tables 1  3
%%% tables
% Table1
% wide table \usepackage{tabularx}
% makecell{}

\begin{table}
%\begin{sidewaystable}[hp]
\centering
\caption{ The 10 candidate genes overexpressed with poor prognosis in \acrshort{hnscc} (ranked by Bonferroni adjusted \textit{P} value) }
\arrayrulecolor[rgb]{0.255,0.255,0.255}
\resizebox{\linewidth}{!}{
%\begin{tabular}{|l|l|l|l|l|l|l|l|c|}
\begin{tabularx}{\textwidth}{|l|p{3.5cm}|l|l|l|l|l|l|c|} % *** capital X or p{2cm}
% p{2cm}
\hline
\multicolumn{1}{|c|}{\multirow{2}{*}{Gene ID}} & \multicolumn{1}{c|}{\multirow{2}{*}{Gene Description}}     & \multicolumn{2}{c|}{Kaplan-Meier survival}                                                                                   & \multicolumn{2}{c|}{Univariate~}                                                        & \multicolumn{2}{c|}{Multivariate}                                                       & \multirow{2}{*}{Remark}                     \\ 
\cline{3-8}
\multicolumn{1}{|c|}{}                         & \multicolumn{1}{c|}{}                                      & \multicolumn{1}{c|}{\textit{P} value}               & \multicolumn{1}{c|}{\begin{tabular}[c]{@{}c@{}}Adjusted \\\textit{P} value\end{tabular}} & \multicolumn{1}{c|}{HR*}                   & 95\% CI                                    & HR*                                        & 95\% CI                                    &                                             \\ 
\hline
DKK1                                           & dickkopf WNT signaling pathway inhibitor 1                 & \num{8.9e-8}                               & 0.001                                                                           & 2.266                                      & 1.666-3.082                                & 2.135                                      & 1.559-2.924                                & **                                          \\ 
\hline
CAMK2N1                                        & calcium/calmodulin-dependent protein kinase II inhibitor 1 & \num{2.9e-7}                               & 0.002                                                                           & 2.101                                      & 1.572-2.809                                & 2.007                                      & 1.490-2.704                                & **                                          \\ 
\hline
STC2                                           & stanniocalcin 2                                            & \num{6.5e-7}                               & 0.004                                                                           & 2.147                                      & 1.578-2.921                                & 2.075                                      & 1.515-2.843                                & **                                          \\ 
\hline
PGK1                                           & phosphoglycerate kinase 1                                  & \num{9.1e-7}                               & 0.006                                                                           & 2.127                                      & 1.563-2.895                                & 2.046                                      & 1.498-2.795                                & **                                          \\ 
\hline
SURF4                                          & surfeit 4                                                  & \num{9.6e-7}                               & 0.006                                                                           & 2.055                                      & 1.531-2.757                                & 2.089                                      & 1.543-2.829                                & 3                                           \\ 
\hline
USP10                                          & ubiquitin specific peptidase 10                            & \num{1.7e-6}                               & 0.012                                                                           & 2.083                                      & 1.532-2.834                                & 2.119                                      & 1.551-2.895                                & **                                          \\ 
\hline
NDFIP1                                         & Nedd4 family interacting protein 1                         & \num{2.6e-6}                               & 0.017                                                                           & 2.031                                      & 1.502-2.746                                & 2.027                                      & 1.483-2.771                                & 6                                           \\ 
\hline
FOXA2                                          & forkhead box A2                                            & \num{2.7e-6}                               & 0.018                                                                           & 1.976                                      & 1.479-2.640                                & 1.914                                      & 1.426-2.569                                & **                                          \\ 
\hline
STIP1                                          & stress-induced-phosphoprotein 1                            & \num{4.3e-6}                              & 0.029                                                                           & 1.958                                      & 1.463-2.621                                & 1.957                                      & 1.451-2.640                                & **                                          \\ 
\hline
DKC1                                           & dyskeratosis congenita 1, dyskerin                         & \num{6.3e-6}                               & 0.042                                                                           & 2.046                                      & 1.490-2.808                                & 1.837                                      & 1.332-2.534                                & **                                          \\ 
\hline
\multicolumn{1}{|l!{\color{black}\vrule}}{}    & \multicolumn{1}{l!{\color{black}\vrule}}{}                 & \multicolumn{1}{l!{\color{black}\vrule}}{} & \multicolumn{1}{l!{\color{black}\vrule}}{}                                      & \multicolumn{1}{l!{\color{black}\vrule}}{} & \multicolumn{1}{l!{\color{black}\vrule}}{} & \multicolumn{1}{l!{\color{black}\vrule}}{} & \multicolumn{1}{l!{\color{black}\vrule}}{} & \multicolumn{1}{l!{\color{black}\vrule}}{}  \\ 
\arrayrulecolor{black}\hline
\multicolumn{9}{|l!{\color{black}\vrule}}{\begin{tabular}[c]{@{}l@{}}Selection criteria:~~\\~Kaplan-Meier Bonferroni adjusted \textit{P} $< 0.05 $~\\~Cox's univariate and multivariate$ HR >= 1.5$ \end{tabular}}                                                                                                                                                                                                                                                              \\ 
\hline
\multicolumn{9}{|l!{\color{black}\vrule}}{* Cox's model: \textit{P} $< 0.001$ }                                                                                                                                                                                                                                                                                                                                                                                                  \\ 
\hline
\multicolumn{9}{|l!{\color{black}\vrule}}{Remark: number of articles related with cancer research; ** as many~}                                                                                                                                                                                                                                                                                                                                                             \\
\hline
\end{tabularx}
}
\arrayrulecolor{black}
\label{table:table1}

%\end{sidewaystable}
\end{table}




%\subsubsection{Table3/legend}
%Table3. The 10 candidate genes overexpressed with better prognosis in HNSCC (ranked by Bonferroni corrected Kaplan-Meier P-value).
\begin{table}[hp]
\centering
\caption{The 10 candidate genes overexpressed with better prognosis in \acrshort{hnscc} (ranked by Bonferroni corrected \textit{P} value) }
\arrayrulecolor[rgb]{0.255,0.255,0.255}
\resizebox{\linewidth}{!}{%
\begin{tabular}{|l|l|l|l|l|l|l|l|c|} 
\hline
\multicolumn{1}{|c|}{\multirow{2}{*}{Gene ID}} & \multicolumn{1}{c|}{\multirow{2}{*}{Gene Description}} & \multicolumn{2}{l|}{Kaplan-Meier survival}                                                                                                                                                & \multicolumn{2}{c|}{Univariate~}                                                                        & \multicolumn{2}{c|}{Multivariate}                                                                       & \multicolumn{1}{l|}{\multirow{2}{*}{Remark}}  \\ 
\cline{3-8}
\multicolumn{1}{|c|}{}                         & \multicolumn{1}{c|}{}                                  & \multicolumn{1}{c!{\color{black}\vrule}}{\textit{P} value}                                  & \multicolumn{1}{c!{\color{black}\vrule}}{\begin{tabular}[c]{@{}c@{}}Adjusted\\~\textit{P} value\end{tabular}} & \multicolumn{1}{c!{\color{black}\vrule}}{HR*}   & \multicolumn{1}{c!{\color{black}\vrule}}{95\% CI}     & \multicolumn{1}{c!{\color{black}\vrule}}{HR*}   & \multicolumn{1}{c!{\color{black}\vrule}}{95\% CI}     & \multicolumn{1}{l|}{}                         \\ 
\cline{1-2}\arrayrulecolor{black}\cline{3-8}\arrayrulecolor[rgb]{0.255,0.255,0.255}\cline{9-9}
ZNF557                                         & zinc finger protein 557                                & \multicolumn{1}{l!{\color{black}\vrule}}{\textcolor[rgb]{0,0,0.471}{\num{8.6e-8}}} & \multicolumn{1}{l!{\color{black}\vrule}}{0.001}                                                      & \multicolumn{1}{l!{\color{black}\vrule}}{0.465} & \multicolumn{1}{l!{\color{black}\vrule}}{0.348-0.619} & \multicolumn{1}{l!{\color{black}\vrule}}{0.499} & \multicolumn{1}{l!{\color{black}\vrule}}{0.372-0.669} & 0                                             \\ 
\cline{1-2}\arrayrulecolor{black}\cline{3-8}\arrayrulecolor[rgb]{0.255,0.255,0.255}\cline{9-9}
ZNF266                                         & zinc finger protein 266                                & \textcolor[rgb]{0,0,0.471}{\num{2.2e-7}}                                           & 0.001                                                                                                & 0.474                                           & 0.355-0.632                                           & 0.453                                           & 0.338-0.607                                           & 1                                             \\ 
\hline
IL19                                           & interleukin 19                                         & \textcolor[rgb]{0,0,0.471}{}\num{3.7e-7}\textcolor[rgb]{0,0,0.471}{}               & 0.002                                                                                                & 0.472                                           & 0.351-0.635                                           & 0.459                                           & 0.340-0.619                                           & 14                                            \\ 
\hline
MYO1H                                          & myosin 1H                                              & \textcolor[rgb]{0,0,0.471}{}\num{3.8e-7}\textcolor[rgb]{0,0,0.471}{}               & 0.003                                                                                                & 0.468                                           & 0.347-0.632                                           & 0.467                                           & 0.344-0.634                                           & 0                                             \\ 
\hline
FCGBP                                          & Fc fragment of IgG binding protein                     & \textcolor[rgb]{0,0,0.471}{}\num{1.2e-6}\textcolor[rgb]{0,0,0.471}{}               & 0.008                                                                                                & 0.484                                           & 0.359-0.653                                           & 0.496                                           & 0.366-0.674                                           & **                                            \\ 
\hline
LOC148709                                      & LncRNA LOC148709                                       & \textcolor[rgb]{0,0,0.471}{\num{1.5e-6}}                                           & 0.010                                                                                                & 0.499                                           & 0.374-0.666                                           & 0.485                                           & 0.361-0.652                                           & 1                                             \\ 
\hline
EVPLL                                          & envoplakin-like protein                                & \textcolor[rgb]{0,0,0.471}{\num{2.0e-6}}                                           & 0.013                                                                                                & 0.490                                           & 0.363-0.661                                           & 0.494                                           & 0.364-0.672                                           & 0                                             \\ 
\hline
PNMA5                                          & paraneoplastic antigen like 5                          & \textcolor[rgb]{0,0,0.471}{\num{2.6e-6}}                                           & 0.017                                                                                                & 0.499                                           & 0.371-0.671                                           & 0.481                                           & 0.357-0.650                                           & 5                                             \\ 
\hline
KIAA1683                                       & new name as IQ Motif Containing N (IQCN)                           & \textcolor[rgb]{0,0,0.471}{\num{3.1e-6}}                                           & 0.020                                                                                                & 0.500                                           & 0.371-0.673                                           & 0.483                                           & 0.356-0.654                                           & 0                                             \\ 
\hline
NPB                                            & neuropeptide B                                         & \textcolor[rgb]{0,0,0.471}{\num{4.0e-6}}                                           & 0.027                                                                                                & 0.460                                           & 0.328-0.646                                           & 0.457                                           & 0.324-0.646                                           & 4                                             \\ 
\hline
                                               &                                                        &                                                                                    &                                                                                                      &                                                 &                                                       &                                                 &                                                       & \multicolumn{1}{l|}{}                         \\ 
\hline
\multicolumn{9}{|l|}{\begin{tabular}[c]{@{}l@{}}Selection criteria:~\\~Kaplan-Meier Bonferroni adjusted \textit{P} value \textless{} 0.05~\\~Cox's univariate and multivariate HR \textgreater{}= 1.5\\* Cox's model: \textit{P} value \textless{} 0.001\\Remark: number off articles related to cancer research; ** as many~\\lncRNA: Long non-coding RNA\end{tabular}}                                                                                                                                                                                                                 \\
\hline
\end{tabular}
}
\arrayrulecolor{black}
\label{table:table3}
\end{table}




%\subsubsection{Figure 3:h}
\begin{figure}[hp]
\centering
\includegraphics[width=14cm]{TCGA_HNSC_Optimal_Overall_allPlot_unKM_P_multiHR-Figure3.pdf} % .png from PvalueplotKM_20genes_Bonf.pdf
\bcaption{Volcano plot of genes under survival analyses.}
{X axis: unadjusted \textit{P} value of Kaplan-Meier survival (-log10 transformed).
Y axis: multivariate hazard ratio from Cox proportional regression model.
Dotted line: significant Bonferroni corrected \textit{P} value. 
\textcolor{red}{Red circles} mark 10 candidate genes, which impact on poor prognosis ($HR>=1.5$). \textcolor{green}{Green circles} mark 10 genes, which affect on better survival ($HR<=0.5$).}
\label{fig:figure3}
\end{figure}
% new figure of volcano plot: TCGA_HNSC_Optimal_Overall_allPlot_unKM_P_multiHR.pdf, -log10(raw KM P-value) vs Cox multivariate HR
% X axis: "frequency" (number of uncorrected P-values less than 0.05 during Kaplan-Meier cutoff finding procedure). X axis: adjusted Kaplan-Meier P-value.

\clearpage


Our top 1 candidate is \acrfull{DKK1}. The Kaplan-Meier curve reveals 227 patients bearing the higher expression of \acrshort{DKK1} were suffered from only 40\% of 5-year \acrshort{os} rate. In comparison, the other 187 patients with lower expression (the cutoff at -0.312(RSEM)) had a better prognosis (adjusted \textit{P} = $0.001$) (see Figure \ref{fig:figure4}(a)).
Figure \ref{fig:figure4}(b)'s cumulative \textit{P} value plot shows that the uncorrected 116 \textit{P} values ($< 0.05$) have been estimated by a serial cut from 125 to 290 persons for grouping the cohort in our cutoff finding procedure (cutofFinder\_func.R, see Figure \ref{fig:figure1}, "HNSCC cohort"). The smallest \textit{P} value (\num{8.9e-8}), when cut on n=187 (45.2\% of 414), has been defined as an optimal \textit{P} value.
Conversely, the most associated gene with better survival is \acrfull{ZNF557}. In Figure \ref{fig:figure4}(c), a Kaplan-Meier curve reveals 264 patients bearing the higher expression of \acrshort{ZNF557} had 55\% of 5-year OS survival rate (adjusted \textit{P} = $0.001$). The cutoff finding procedure (cutofFinder\_func.R) generated cumulative \textit{P} value plot in Figure \ref{fig:figure4}(d). The 166 uncorrected \textit{P} values were estimated by a serial cut from 125 to 290 for grouping the cohort. The smallest \textit{P} value (\num{8.6e-8}), when cut on n=150 (36.2\% of total cohort 414), has been defined as an optimal \textit{P} value with a cutoff value -0.511(RSEM) of \acrshort{rnaseq}.
%The second candidate, which improving patient survival, is ZNF266.
%\subsubsection{Tables} in main article

%\subsubsection{Figure 4:h}
\begin{figure}[hp]
\centering
\includegraphics[width=15cm]{Figure4.pdf}
\bcaption{Kaplan-Meier survival analyses, by cutoff finding.}
{(a) Kaplan-Meier plot of DKK1 under optimal \textit{P} value, and (b) the cutoff is derived from cumulative \textit{P} value plot of DKK1. (c) Kaplan-Meier plot of ZNF557 under optimal \textit{P} value, and (d) the cutoff is derived from cumulative \textit{P} value plot of ZNF557.}
\label{fig:figure4}
\end{figure}

\clearpage

Table \ref{table:table1} shows ten overexpressed genes are associated with poor prognosis in \acrshort{hnscc}, ranked by adjusted Kaplan-Meier \textit{P} value. 
We found their Cox's univariate and multivariate HR are all greater than 1.837. 
%There were few published articles of \acrfull{SURF4} and \acrfull{NDFIP1} (\acrshort{NEDD4}: \acrlong{NEDD4}), which were related to cancer research.
In Table \ref{table:table2},
after adjustment of confounders, \acrshort{DKK1} overexpression is the independent prognostic factor (multivariate HR 2.135 [95\% CI: 1.559-2.924, \textit{P} $<$ 0.001]), as well as clinical T stage (HR 1.978 [95\% CI: 1.046-3.737, \textit{P} = 0.036]) and surgical margins status (HR 1.601 [95\% CI: 1.159-2.211, \textit{P} = 0.004]). 
Older age (more than 65) also worse the survival (HR 1.462 [95\% CI: 1.078-1.983, \textit{P} = 0.015]). 
The M stage could be ignored in this cohort due to only 3 out of 414 patients with distant metastasis.

%\subsubsection{Table2/legend}
%Table 2. Univariate/Multivariate Cox's proportional hazards regression analyses on OS time of DKK1 gene expression in HNSCC.
%(P-value Significant codes is denoted as  0.01 mark as *; 0.001 mark as **; if $<$0.001  mark as ***)
\begin{table}[hp]
\centering
\caption{Univariate/multivariate Cox's proportional hazards regression analyses on OS time of DKK1 gene expression in \acrshort{hnscc}}
\arrayrulecolor[rgb]{0.255,0.255,0.255}
\resizebox{\linewidth}{!}{%
\begin{tabular}{|l|l|l|l|l|l|l|l|} 
\arrayrulecolor{black}\cline{1-2}\arrayrulecolor[rgb]{0.255,0.255,0.255}\cline{3-8}
\multicolumn{2}{|c!{\color{black}\vrule}}{\multirow{2}{*}{Features}}                                                          & \multicolumn{3}{c|}{Univariate}                                                                                                                                                                                                                & \multicolumn{3}{c|}{Multivariate}                                                                                                                                                                                                               \\ 
\cline{3-8}
\multicolumn{2}{|c!{\color{black}\vrule}}{}                                                                                   & \multicolumn{1}{c!{\color{black}\vrule}}{HR}                                   & \multicolumn{1}{c!{\color{black}\vrule}}{CI95\%}                              & \multicolumn{1}{c!{\color{black}\vrule}}{\textit{P} value}                             & \multicolumn{1}{c!{\color{black}\vrule}}{HR}                                   & \multicolumn{1}{c!{\color{black}\vrule}}{CI95\%}                              & \multicolumn{1}{c!{\color{black}\vrule}}{\textit{P} value}                              \\ 
\arrayrulecolor{black}\hline
\multirow{2}{*}{Gender}                 & \multicolumn{1}{l!{\color{black}\vrule}}{{\cellcolor[rgb]{0.62,0.812,0.878}}Female} & \multicolumn{1}{l!{\color{black}\vrule}}{{\cellcolor[rgb]{0.62,0.812,0.878}}1} & \multicolumn{1}{l!{\color{black}\vrule}}{{\cellcolor[rgb]{0.62,0.812,0.878}}} & \multicolumn{1}{l!{\color{black}\vrule}}{{\cellcolor[rgb]{0.62,0.812,0.878}}} & \multicolumn{1}{l!{\color{black}\vrule}}{{\cellcolor[rgb]{0.62,0.812,0.878}}1} & \multicolumn{1}{l!{\color{black}\vrule}}{{\cellcolor[rgb]{0.62,0.812,0.878}}} & \multicolumn{1}{l!{\color{black}\vrule}}{{\cellcolor[rgb]{0.62,0.812,0.878}}}  \\ 
\cline{2-8}
                                        & Male                                                                                & 1.157                                                                          & 0.843-1.587                                                                   & 0.367                                                                         & 1.178                                                                          & 0.841-1.650                                                                   & 0.342                                                                          \\ 
\arrayrulecolor[rgb]{0.255,0.255,0.255}\hline
\multirow{2}{*}{Age at diagnosis}       & {\cellcolor[rgb]{0.62,0.812,0.878}}$<=65y$                                          & {\cellcolor[rgb]{0.62,0.812,0.878}}1                                           & {\cellcolor[rgb]{0.62,0.812,0.878}}                                           & {\cellcolor[rgb]{0.62,0.812,0.878}}                                           & {\cellcolor[rgb]{0.62,0.812,0.878}}1                                           & {\cellcolor[rgb]{0.62,0.812,0.878}}                                           & {\cellcolor[rgb]{0.62,0.812,0.878}}                                            \\ 
\cline{2-8}
                                        & $>65y$                                                                              & 1.329                                                                          & 0.990-1.784                                                                   & 0.058                                                                         & 1.462                                                                          & 1.078-1.983                                                                   & \textcolor{red}{0.015}                                                         \\ 
\hline
\multirow{2}{*}{Clinical T Status}      & {\cellcolor[rgb]{0.62,0.812,0.878}}T1+T2                                            & {\cellcolor[rgb]{0.62,0.812,0.878}}1                                           & {\cellcolor[rgb]{0.62,0.812,0.878}}                                           & {\cellcolor[rgb]{0.62,0.812,0.878}}                                           & {\cellcolor[rgb]{0.62,0.812,0.878}}1                                           & {\cellcolor[rgb]{0.62,0.812,0.878}}                                           & {\cellcolor[rgb]{0.62,0.812,0.878}}                                            \\ 
\cline{2-8}
                                        & T3+T4                                                                               & 1.409                                                                          & 1.028-1.931                                                                   & \textcolor{red}{0.033}                                                        & 1.978                                                                          & 1.046-3.737                                                                   & \textcolor{red}{0.036}                                                         \\ 
\hline
\multirow{2}{*}{Clinical N Status}      & {\cellcolor[rgb]{0.62,0.812,0.878}}N0                                               & {\cellcolor[rgb]{0.62,0.812,0.878}}1                                           & {\cellcolor[rgb]{0.62,0.812,0.878}}                                           & {\cellcolor[rgb]{0.62,0.812,0.878}}                                           & {\cellcolor[rgb]{0.62,0.812,0.878}}1                                           & {\cellcolor[rgb]{0.62,0.812,0.878}}                                           & {\cellcolor[rgb]{0.62,0.812,0.878}}                                            \\ 
\cline{2-8}
                                        & N1-3                                                                                & 1.185                                                                          & 0.890-1.577                                                                   & 0.246                                                                         & 1.149                                                                          & 0.805-1.640                                                                   & 0.445                                                                          \\ 
\hline
\multirow{2}{*}{Clinical M Status}      & {\cellcolor[rgb]{0.62,0.812,0.878}}M0                                               & {\cellcolor[rgb]{0.62,0.812,0.878}}1                                           & {\cellcolor[rgb]{0.62,0.812,0.878}}                                           & {\cellcolor[rgb]{0.62,0.812,0.878}}                                           & {\cellcolor[rgb]{0.62,0.812,0.878}}1                                           & {\cellcolor[rgb]{0.62,0.812,0.878}}                                           & {\cellcolor[rgb]{0.62,0.812,0.878}}                                            \\ 
\cline{2-8}
                                        & M1                                                                                  & 4.097                                                                          & 1.009-16.64                                                                   & \textcolor{red}{0.049}                                                        & 6.513                                                                          & 1.415-29.96                                                                   & \textcolor{red}{0.016}                                                         \\ 
\hline
\multirow{2}{*}{Clinical Stage}         & {\cellcolor[rgb]{0.62,0.812,0.878}}Stage I+II                                       & {\cellcolor[rgb]{0.62,0.812,0.878}}1                                           & {\cellcolor[rgb]{0.62,0.812,0.878}}                                           & {\cellcolor[rgb]{0.62,0.812,0.878}}                                           & {\cellcolor[rgb]{0.62,0.812,0.878}}1                                           & {\cellcolor[rgb]{0.62,0.812,0.878}}                                           & {\cellcolor[rgb]{0.62,0.812,0.878}}                                            \\ 
\cline{2-8}
                                        & Stage III+IV                                                                        & 1.245                                                                          & 0.882-1.759                                                                   & 0.213                                                                         & 0.597                                                                          & 0.277-1.287                                                                   & 0.188                                                                          \\ 
\hline
\multirow{2}{*}{Surgical Margin status} & {\cellcolor[rgb]{0.62,0.812,0.878}}Negative                                         & {\cellcolor[rgb]{0.62,0.812,0.878}}1                                           & {\cellcolor[rgb]{0.62,0.812,0.878}}                                           & {\cellcolor[rgb]{0.62,0.812,0.878}}                                           & {\cellcolor[rgb]{0.62,0.812,0.878}}1                                           & {\cellcolor[rgb]{0.62,0.812,0.878}}                                           & {\cellcolor[rgb]{0.62,0.812,0.878}}                                            \\ 
\cline{2-8}
                                        & Positive                                                                            & 1.591                                                                          & 1.155-2.191                                                                   & \textcolor{red}{0.004}                                                        & 1.601                                                                          & 1.159-2.211                                                                   & \textcolor{red}{0.004}                                                         \\ 
\hline
\multirow{2}{*}{Tobacco Exposure}       & {\cellcolor[rgb]{0.62,0.812,0.878}}Low                                              & {\cellcolor[rgb]{0.62,0.812,0.878}}1                                           & {\cellcolor[rgb]{0.62,0.812,0.878}}                                           & {\cellcolor[rgb]{0.62,0.812,0.878}}                                           & {\cellcolor[rgb]{0.62,0.812,0.878}}1                                           & {\cellcolor[rgb]{0.62,0.812,0.878}}                                           & {\cellcolor[rgb]{0.62,0.812,0.878}}                                            \\ 
\cline{2-8}
                                        & High                                                                                & 1.364                                                                          & 1.008-1.844                                                                   & \textcolor{red}{0.044}                                                        & 1.302                                                                          & 0.943-1.797                                                                   & 0.109                                                                          \\ 
\hline
\multirow{2}{*}{RNA-Seq}                & {\cellcolor[rgb]{0.62,0.812,0.878}}Low                                              & {\cellcolor[rgb]{0.62,0.812,0.878}}1                                           & {\cellcolor[rgb]{0.62,0.812,0.878}}                                           & {\cellcolor[rgb]{0.62,0.812,0.878}}                                           & {\cellcolor[rgb]{0.62,0.812,0.878}}1                                           & {\cellcolor[rgb]{0.62,0.812,0.878}}                                           & {\cellcolor[rgb]{0.62,0.812,0.878}}                                            \\ 
\cline{2-8}
                                        & High                                                                                & 2.266                                                                          & 1.666-3.082                                                                   & \multicolumn{1}{c|}{\textcolor{red}{***}}                                     & 2.135                                                                          & 1.559-2.924                                                                   & \multicolumn{1}{c|}{\textcolor{red}{***}}                                      \\ 
\hline
\multicolumn{8}{|l|}{}                                                                                                                                                                                                                                                                                                                                                                                                                                                                                                                                                                                                           \\ 
\hline
\multicolumn{8}{|l|}{(\textit{P} value significant codes is denoted: $red<0.05$; *** $<0.001$~~)}                                                                                                                                                                                                                                                                                                                                                                                                                                                                                                                                           \\
\hline
\end{tabular}
}
\arrayrulecolor{black}
\label{table:table2}
\end{table}




In Table \ref{table:table3}, the other ten overexpressed genes have been found in better prognosis of \acrshort{hnscc} patients. Cox's univariate and multivariate HR is under 0.5.
In Table \ref{table:table4},
after adjustment of confounders, prognosis is influenced by advance clinical T Status (HR 1.961 [95\% CI: 1.035-3.714, \textit{P} = 0.039] ), positive surgical margin involvement (HR 1.631 [95\% CI: 1.18-2.254, \textit{P} = 0.003]) , and higher tobacco exposure (HR 1.453 [95\% CI: 1.055-2.000, \textit{P} = 0.022]).
Overexpressed \acrshort{ZNF557} gene could has a protective influencce on prognosis (HR 0.499 [95\% CI: 0.372-0.669, \textit{P} $<$ 0.001]).


In summary, those 20 candidate biomarkers, clinical T stage, and surgical margin are independent prognosis factors in \acrshort{hnscc}.
Thus, the prognosis model with coefficients is established from \acrshort{tcga} \acrshort{hnscc} cohort. The important input $X_1...X_n$ should be patients' features: age, specific 20 gene expressions, clinical T stage, and surgical margin.
%\\[0.5cm]




%\subsubsection{Table4/legend}
\begin{table}[!hp]
\centering
\caption{Univariate/multivariate Cox's proportional hazards regression analyses on OS time of ZNF557 gene expression in \acrshort{hnscc}}
\arrayrulecolor[rgb]{0.255,0.255,0.255}
\resizebox{\linewidth}{!}{%
\begin{tabular}{|l|l|l|l|l|l|l|l|} 
\arrayrulecolor{black}\cline{1-2}\arrayrulecolor[rgb]{0.255,0.255,0.255}\cline{3-8}
\multicolumn{2}{|c!{\color{black}\vrule}}{\multirow{2}{*}{Features}}                                                          & \multicolumn{3}{c|}{Univariate}                                                                                                                                                                                                                & \multicolumn{3}{c|}{Multivariate}                                                                                                                                                                                                               \\ 
\cline{3-8}
\multicolumn{2}{|c!{\color{black}\vrule}}{}                                                                                   & \multicolumn{1}{c!{\color{black}\vrule}}{HR}                                   & \multicolumn{1}{c!{\color{black}\vrule}}{CI95\%}                              & \multicolumn{1}{c!{\color{black}\vrule}}{\textit{P} value}                             & \multicolumn{1}{c!{\color{black}\vrule}}{HR}                                   & \multicolumn{1}{c!{\color{black}\vrule}}{CI95\%}                              & \multicolumn{1}{c!{\color{black}\vrule}}{\textit{P} value}                              \\ 
\arrayrulecolor{black}\hline
\multirow{2}{*}{Gender}                 & \multicolumn{1}{l!{\color{black}\vrule}}{{\cellcolor[rgb]{0.62,0.812,0.878}}Female} & \multicolumn{1}{l!{\color{black}\vrule}}{{\cellcolor[rgb]{0.62,0.812,0.878}}1} & \multicolumn{1}{l!{\color{black}\vrule}}{{\cellcolor[rgb]{0.62,0.812,0.878}}} & \multicolumn{1}{l!{\color{black}\vrule}}{{\cellcolor[rgb]{0.62,0.812,0.878}}} & \multicolumn{1}{l!{\color{black}\vrule}}{{\cellcolor[rgb]{0.62,0.812,0.878}}1} & \multicolumn{1}{l!{\color{black}\vrule}}{{\cellcolor[rgb]{0.62,0.812,0.878}}} & \multicolumn{1}{l!{\color{black}\vrule}}{{\cellcolor[rgb]{0.62,0.812,0.878}}}  \\ 
\cline{2-8}
                                        & Male                                                                                & 1.157                                                                          & 0.843-1.587                                                                   & 0.367                                                                         & 1.163                                                                          & 0.833-1.625                                                                   & 0.375                                                                          \\ 
\arrayrulecolor[rgb]{0.255,0.255,0.255}\hline
\multirow{2}{*}{Age at diagnosis}       & {\cellcolor[rgb]{0.62,0.812,0.878}}$<=65y$                                          & {\cellcolor[rgb]{0.62,0.812,0.878}}1                                           & {\cellcolor[rgb]{0.62,0.812,0.878}}                                           & {\cellcolor[rgb]{0.62,0.812,0.878}}                                           & {\cellcolor[rgb]{0.62,0.812,0.878}}1                                           & {\cellcolor[rgb]{0.62,0.812,0.878}}                                           & {\cellcolor[rgb]{0.62,0.812,0.878}}                                            \\ 
\cline{2-8}
                                        & $>65y$                                                                              & 1.329                                                                          & 0.990-1.784                                                                   & 0.058                                                                         & 1.328                                                                          & 0.976-1.808                                                                   & 0.071                                                                          \\ 
\hline
\multirow{2}{*}{Clinical T Status}      & {\cellcolor[rgb]{0.62,0.812,0.878}}T1+T2                                            & {\cellcolor[rgb]{0.62,0.812,0.878}}1                                           & {\cellcolor[rgb]{0.62,0.812,0.878}}                                           & {\cellcolor[rgb]{0.62,0.812,0.878}}                                           & {\cellcolor[rgb]{0.62,0.812,0.878}}1                                           & {\cellcolor[rgb]{0.62,0.812,0.878}}                                           & {\cellcolor[rgb]{0.62,0.812,0.878}}                                            \\ 
\cline{2-8}
                                        & T3+T4                                                                               & 1.409                                                                          & 1.028-1.931                                                                   & \textcolor[rgb]{1,0.149,0}{0.033}                                             & 1.961                                                                          & 1.035-3.714                                                                   & \textcolor[rgb]{1,0.149,0}{0.039}                                              \\ 
\hline
\multirow{2}{*}{Clinical N Status}      & {\cellcolor[rgb]{0.62,0.812,0.878}}N0                                               & {\cellcolor[rgb]{0.62,0.812,0.878}}1                                           & {\cellcolor[rgb]{0.62,0.812,0.878}}                                           & {\cellcolor[rgb]{0.62,0.812,0.878}}                                           & {\cellcolor[rgb]{0.62,0.812,0.878}}1                                           & {\cellcolor[rgb]{0.62,0.812,0.878}}                                           & {\cellcolor[rgb]{0.62,0.812,0.878}}                                            \\ 
\cline{2-8}
                                        & N1-3                                                                                & 1.185                                                                          & 0.890-1.577                                                                   & 0.246                                                                         & 1.179                                                                          & 0.824-1.686                                                                   & 0.367                                                                          \\ 
\hline
\multirow{2}{*}{Clinical M Status}      & {\cellcolor[rgb]{0.62,0.812,0.878}}M0                                               & {\cellcolor[rgb]{0.62,0.812,0.878}}1                                           & {\cellcolor[rgb]{0.62,0.812,0.878}}                                           & {\cellcolor[rgb]{0.62,0.812,0.878}}                                           & {\cellcolor[rgb]{0.62,0.812,0.878}}1                                           & {\cellcolor[rgb]{0.62,0.812,0.878}}                                           & {\cellcolor[rgb]{0.62,0.812,0.878}}                                            \\ 
\cline{2-8}
                                        & M1                                                                                  & 4.097                                                                          & 1.009-16.64                                                                   & \textcolor[rgb]{1,0.149,0}{0.049}                                             & 8.478                                                                          & 1.847-38.92                                                                   & \textcolor[rgb]{1,0.149,0}{0.006}                                              \\ 
\hline
\multirow{2}{*}{Clinical Stage}         & {\cellcolor[rgb]{0.62,0.812,0.878}}Stage I+II                                       & {\cellcolor[rgb]{0.62,0.812,0.878}}1                                           & {\cellcolor[rgb]{0.62,0.812,0.878}}                                           & {\cellcolor[rgb]{0.62,0.812,0.878}}                                           & {\cellcolor[rgb]{0.62,0.812,0.878}}1                                           & {\cellcolor[rgb]{0.62,0.812,0.878}}                                           & {\cellcolor[rgb]{0.62,0.812,0.878}}                                            \\ 
\cline{2-8}
                                        & Stage III+IV                                                                        & 1.245                                                                          & 0.882-1.759                                                                   & 0.213                                                                         & 0.512                                                                          & 0.239-1.096                                                                   & 0.085                                                                          \\ 
\hline
\multirow{2}{*}{Surgical Margin status} & {\cellcolor[rgb]{0.62,0.812,0.878}}Negative                                         & {\cellcolor[rgb]{0.62,0.812,0.878}}1                                           & {\cellcolor[rgb]{0.62,0.812,0.878}}                                           & {\cellcolor[rgb]{0.62,0.812,0.878}}                                           & {\cellcolor[rgb]{0.62,0.812,0.878}}1                                           & {\cellcolor[rgb]{0.62,0.812,0.878}}                                           & {\cellcolor[rgb]{0.62,0.812,0.878}}                                            \\ 
\cline{2-8}
                                        & Positive                                                                            & 1.591                                                                          & 1.155-2.191                                                                   & \textcolor[rgb]{1,0.149,0}{0.004}                                             & 1.631                                                                          & 1.180-2.254                                                                   & \textcolor[rgb]{1,0.149,0}{0.003}                                              \\ 
\hline
\multirow{2}{*}{Tobacco Exposure}       & {\cellcolor[rgb]{0.62,0.812,0.878}}Low                                              & {\cellcolor[rgb]{0.62,0.812,0.878}}1                                           & {\cellcolor[rgb]{0.62,0.812,0.878}}                                           & {\cellcolor[rgb]{0.62,0.812,0.878}}                                           & {\cellcolor[rgb]{0.62,0.812,0.878}}1                                           & {\cellcolor[rgb]{0.62,0.812,0.878}}                                           & {\cellcolor[rgb]{0.62,0.812,0.878}}                                            \\ 
\cline{2-8}
                                        & High                                                                                & 1.364                                                                          & 1.008-1.844                                                                   & \textcolor[rgb]{1,0.149,0}{0.044}                                             & 1.453                                                                          & 1.055-2.000                                                                   & \textcolor[rgb]{1,0.149,0}{0.022}                                              \\ 
\hline
\multirow{2}{*}{RNA-Seq}                & {\cellcolor[rgb]{0.62,0.812,0.878}}Low                                              & {\cellcolor[rgb]{0.62,0.812,0.878}}1                                           & {\cellcolor[rgb]{0.62,0.812,0.878}}                                           & {\cellcolor[rgb]{0.62,0.812,0.878}}                                           & {\cellcolor[rgb]{0.62,0.812,0.878}}1                                           & {\cellcolor[rgb]{0.62,0.812,0.878}}                                           & {\cellcolor[rgb]{0.62,0.812,0.878}}                                            \\ 
\cline{2-8}
                                        & High                                                                                & 0.465                                                                          & 0.348-0.619                                                                   & \multicolumn{1}{c|}{\textcolor[rgb]{1,0.149,0}{***}}                          & 0.499                                                                          & 0.372-0.669                                                                   & \multicolumn{1}{c|}{\textcolor[rgb]{1,0.149,0}{***}}                           \\ 
\hline
\multicolumn{8}{|l|}{}                                                                                                                                                                                                                                                                                                                                                                                                                                                                                                                                                                                                           \\ 
\hline
\multicolumn{8}{|l|}{(\textit{P} value significant codes is denoted: $red<0.05$; *** $<0.001$~~)}                                                                                                                                                                                                                                                                                                                                                                                                                                                                                                                                           \\
\hline
\end{tabular}
}
\arrayrulecolor{black}
\label{table:table4}
\end{table}



% end of result



%Bulleted lists look like this:
%\begin{itemize}
%\item	First bullet;
%\item	Second bullet;
%\item	Third bullet.
%\end{itemize}

%Numbered lists can be added as follows:
%\begin{enumerate}
%\item	First item; 
%\item	Second item;
%\item	Third item.
%\end{enumerate}



%\subsection{Figures, Tables and Schemes}

%All figures and tables should be cited in the main text as Figure~\ref{fig1}, Table~\ref{tab1}, etc.

%\begin{figure}[H]
%\includegraphics[width=10.5 cm]{Definitions/logo-mdpi}
%\caption{This is a figure. Schemes follow the same formatting. If there are multiple panels, they should be listed as: (\textbf{a}) Description of what is contained in the first panel. (\textbf{b}) Description of what is contained in the second panel. Figures should be placed in the main text near to the first time they are cited. A caption on a single line should be centered.\label{fig1}}
%\end{figure}   

% The MDPI table float is called specialtable
%\begin{specialtable}[H] 
%\caption{This is a table caption. Tables %should be placed in the main text near to %the first time they are~cited.\label{tab1}}
%%% \tablesize{} %% You can specify the fontsize here, e.g., \tablesize{\footnotesize}. If commented out \small will be used.
%\begin{tabular}{ccc}
%\toprule
%\textbf{Title 1}	& \textbf{Title 2}	& %\textbf{Title 3}\\
%\midrule
%Entry 1		& Data			& Data\\
%Entry 2		& Data			& Data\\
%\bottomrule
%\end{tabular}
%\end{specialtable}

%\begin{listing}[H]
%\caption{Title of the listing}
%\rule{\columnwidth}{1pt}
%\raggedright Text of the listing. In font size footnotesize, small, or normalsize. Preferred format: left aligned and single spaced. Preferred border format: top border line and bottom border line.
%\rule{\columnwidth}{1pt}
%\end{listing}


%\subsection{Formatting of Mathematical Components}

%This is the example 1 of equation:

%\begin{equation}
%a = 1,
%\end{equation}

%the text following an equation need not be a new paragraph. Please punctuate equations as regular text.
%% If the documentclass option "submit" is chosen, please insert a blank line before and after any math environment (equation and eqnarray environments). This ensures correct linenumbering. The blank line should be removed when the documentclass option is changed to "accept" because the text following an equation should not be a new paragraph.




\end{paracol}
\nointerlineskip


%\begin{equation}
%a = b + c + d + e + f + g + h + i + j + k + l + m + n + o + p + q + r + s + t + u + v + w + x + y + z
%\end{equation}

% Example of a figure that spans the whole page width (the commands \widefigure and \begin{paracol}{2}, \linenumbers, and\switchcolumn need to be present). The same concept works for tables, too.
  
\begin{paracol}{2}
\linenumbers
\switchcolumn

%Please punctuate equations as regular text. Theorem-type environments (including propositions, lemmas, corollaries etc.) can be formatted as follows:
%% Example of a theorem:
%\begin{Theorem}
%Example text of a theorem.
%\end{Theorem}


%% Example of a proof:
%\begin{proof}[Proof of Theorem 1]
%Text of the proof. Note that the phrase ``of Theorem 1'' is optional if it is clear which theorem is being referred to.
%\end{proof}

%%%%%%%%%%%%%%%%%%%%%%%%%%%%%%%%%%%%%%%%%%%%%%%%%%%%%%%%%%%%%%%%%%%%%%%%%%%%%%%
\section{Discussion}

%%%%
\subsection{Feature selection for survival modeling} % Even there are many $X_1 ... X_n$ physical and social features of patients available for survival modelling in the TCGA.
% TCGAbiolinks
%alcohol consumption
Besides ethnicity, age, gender, \acrshort{tnm} stage, radiation therapy, chemotherapy, and targeted therapy, the comprehensive adversely prognostic features in \acrshort{hnscc} should also include tobacco exposure, \acrshort{egfr} amplification, \acrfull{hpv} status, positive/close surgical margin ($<5 mm$), \acrfull{ene}, \acrfull{lvsi}, \acrfull{pni}, \acrfull{doi} ($>5 mm$), as well as metastatic \acrfull{lnd}\cite{Cheraghlou2018}, and \acrfull{wpoi5}, which is defined as tumor dispersion (1 mm apart between tumor satellites) or positive \acrshort{pni}/\acrshort{lvsi}\cite{Amin2017}.
The features of \acrshort{doi}, \acrshort{lnd}, and tumor dispersion are not available on the \acrshort{tcga} dataset. The Brandwein-Gensler's risk model (lymphocytic host response, \acrshort{wpoi5}, and \acrshort{pni})\cite{Brandwein-Gensler2010}\cite{Sinha2018} has been suggested to be routinely performed on pathological examination. In previous reports of \acrshort{hnscc}, the loco-regional failure will be high when the initial frozen section has a positive/close surgical margin, and even the final margin revision revealed negative\cite{Bulbul2019b}.
According to Table \ref{table:table2} and Table \ref{table:table4} in our study, the positive surgical margin yields a hazard ratio greater than 1.6 to influence on patient's \acrshort{os}.
It is suggested by authors \cite{Scholl1986}\cite{Sutton2003}\cite{Shaw2004}\cite{Guillemaud2010a}\cite{Patel2010}\cite{Kuriakose2017}\cite{Shapiro2017}\cite{Saidak2018}\cite{Miguelanez-Medran2019}\cite{Saidak2019} that the reason of positive/close surgical margin is possibly due to tumor aggressiveness or dispersion (WPOI-5) instead of iatrogenic reason of surgery. The surgical margin status has also been suggested as an independent surrogate for tumor dispersion in the \acrshort{hnscc} study. Thus, we selected common clinicopathological features in the current biomarker discovery, including gender, age, clinical T, clinical N, clinical M, surgical margin status, and tobacco exposure, to adjust confounders (details description at Materials and Methods section).


%%%

\subsection{Validation by web-based tools}
%注意 強調GSE2837: 
The prognostic
significance of the candidate genes has been validated by an independent patient cohort.

%\subsubsection*{GSE2837} % non-TCGA
% from Anwser2-2
%by GEO GSE2837 HNSCC dataset (PrognoScan)
After the discovery with our workflow, we validated the suggested 20 biomarkers in the other \acrshort{hnscc} cohort.
We analysed GSE2837 dataset (NCBI GEO database\cite{Chung2006}, HNSCC cohort has 28 participants) from PrognoScan (availalbe at http://dna00.bio.kyutech.ac.jp/\\PrognoScan/)\cite{Mizuno2009a}.
%??MD Anderson (40 cases) SurvNet: https://bioinformatics.mdanderson.org/SurvNet/
%https://www.ncbi.nlm.nih.gov/geo/query/acc.cgi?acc=GSE2837 \cite{Chung2006}
The GSE2837 carries out % VUMC, VAMC, UTMDACC (1992-2005)
microarray gene expression (by Affymetrix X3P chips U133\_X3P) and relapse-free survival (RFS) data, instead of overall survival in TCGA dataset.
The survival significance (Kaplan Meier \textit{P} value) is in the following probes of 10 genes:\\
1) poor prognosis: CAMK2N1 (0.048214), PGK1 (0.009978), SURF4 (0.023127), USP10 (0.017768), NDFIP1 (0.022758), FOXA2 (0.001587);\\ % FOXA2 (0.038125)
2) better prognosis: IL19 (0.049731), FCGBP (0.005658), KIAA1683 (IQCN, 0.005886), NPB (0.014177);\\
%17 out of 20 candidates
%?DKK1 (0.253635), 
%Nevertheless, PrognoScan has group separation cut by a skewed manner, and the GSE2837 has far fewer participants than the TCGA cohort.
%DKK1, CAMK2N1, STC2, PGK1, SURF4, USP10, NDFIP1, FOXA2, STIP1, DKC1;
%ZNF557, ZNF266, IL19, MYO1H, FCGBP, LOC148709, EVPLL, PNMA5, KIAA1683 (IQCN), NPB
%Those 11 genes achieve similar positive and negative prognostic effect comparable with our proposed candidate genes. 
The 10 out of 20 candidates are confirmed by a comparative study using the GSE2837 dataset other than the TCGA cohort. Please see Kaplan-Meier plots in Supplementary Figure S1.%\ref{fig:fig_GSE2837}.



%
We also utilized another approach of web-based tools (TCGA) to analyze our candidates.
%\subsubsection*{SurvExpress} % TCGA
% 2021/01/23 pdf x 12 from SurvExpress
SurvExpress\cite{Aguirre-Gamboa2013} is one of the web-based tools for Biomarker comparison and validation of survival gene expression data. Using their TCGA HNSCC cohort (June 2016, n=502), the survival significance (Kaplan Meier \textit{P} value) shows in 12 genes:\\ % SurvExpress 有類似的結果:
1) poor prognosis: DKK1 (0.00005), CAMK2N1 (0.000006), STC2 (0.001), PGK1 (0.01), SURF4 (0.002), USP10 (0.002), NDFIP1 (0.000008), STIP1 (0.001), DKC1 (0.01);\\
%CAMK2N1 (0.048214), PGK1 (0.009978), SURF4 (0.023127), USP10 (0.017768), NDFIP1 (0.022758), FOXA2 (0.001587); % FOXA2 (0.038125)
2) better prognosis: ZNF557 (0.0007), ZNF266 (0.00008), FCGBP (0.001).\\
%IL19 (0.049731), FCGBP (0.005658), KIAA1683 (IQCN, 0.005886), NPB (0.014177);
%17 out of 20 candidates
%?DKK1 (0.253635),
The 12 out of 20 candidate genes are confirmed by a comparative study using SurvExpress with the TCGA cohort. Please see Supplementary Figure S2.%\ref{fig:fig_SurvExpress}.
% moved to supplementary file: Supplementary_figures.tex


% % SurvExpress 沒有類似的結果是因為:
Nevertheless, the survival prediction could not be found in FOXA2, IL19, MYO1H, LOC148709, EVPLL, PNMA5, IQCN, KIAA1683, NPB using SurvExpress. 
Our workflow has the advantage to find an optimal cutpoint of that \acrshort{rna} expression data to maximize candidate mining coverage.
%因為 cutoff 不適當,這也就是我們 pvalueTex \textit{P} 的設計目的與強項

 
%\subsubsection{HPA} %still TCGA's RNA-Seq
The Human Protein Atlas project (HPA) has proteomics analysis based on 26,941 antibodies targeting 17,165 unique proteins. The HPA's Pathology Atlas analyzes each protein in patients, using \acrshort{ihc} analysis based on tissue microarrays (TMAs) adopted from TCGA. Kaplan-Meier survival analyses are based on RNA-Seq expression levels of human genes in HNSCC tissue and the clinical outcome.
All transcriptomics data has been retrieved from the TCGA, and all proteomics data has been generated in-house using the same antibodies.
Our 9 candidates (e.g., DKK1, CAMK2N1, STC2, PGK1, SURF4, USP10, NDFIP1, STIP1, and DKC1) are also on the list of unfavorable prognostic genes for HNSCC from \acrfull{hpa} (available at https://www.proteinatlas.org/humanproteome/pathology/head+and+neck+cancer, Version: 20.0 updated: 2020-11-19). The ZNF557, ZNF266, and FCGBP are on the list of favorable prognostic genes as well.
Moreover, overexpression of ubiquitin-specific peptidase 10 (USP10), one of the autophagy-related genes, has a poor prognostic impact on HNSCC, which has been proved at mRNA and protein level (HPA database, available at https://www.proteinatlas.org/ENSG00000103194-USP10/pathology/head+and+neck+\\cancer)\cite{Ren2020}.
% 13 ARGs (GABARAPL1, ITGA3, USP10, ST13, MAPK9, PRKN, FADD, IKBKB, ITPR1, TP73, MAP2K7, CDKN2A, and EEF2K) with prognostic value were identified in HNSCC patients
%found at Human Protein Atlas dataset 
(Please see Supplementary Figure S3)%\ref{fig:fig_HPA_USP10})
% moved to Supplementary



% Version: 3.0.2 Date: 2015/06/11 SurvNet
%Preloaded TCGA  data is also available for analysis.
%DKK1 has the P-value 1.99328e-07 of multivariable Cox proportional hazards regression model, which was reported by network-based biomarkers research tools (SurvNet, accessed at http://bioinformatics.mdanderson.org/main/SurvNet; saved as supplementary file SurvNet\_HNSCC\_result.tsv)\cite{Li2012a}. %gene regulatory or protein interaction network
% cited by 10 articles
% he P-values pi from a univariable Cox proportional hazards regression model, which quantifies how significantly the molecular profiling data of the gene correlate with the patient survival data. 
% SurvNet also calculates the mutivariable Cox P-values for each subnetwork (a group of genes) to validate their clinical utility.
%The network files are in a '.dot' format that can be visualized by GraphViz (http://www.graphviz.org)



%%%
\subsection{The candidates shown in literature review}

%%
%\subsection*{Review articles}
Since the discovery of workflow, we performed the literature review by Embase/Pubmed.% to find the evidence convincing the suggested biomarkers in cancer research. 
The 10 out of 20 candidates has also been suggested by published studies using TCGA cohort, in-house cohort, or experiments in vitro and in vivo.

%Embase searching; %https://www.ncbi.nlm.nih.gov/research/pubtator/?view=docsum&query=CAMK2N1%20head%20and%20neck%20cancer
%through the PubMed searching, the remark on table 1 and table 2 presents the cancer research articles related to our candidate genes.
%PubTator Central\cite{Wei2019} is a useful Pubmed text miner

%%%%%% bad guy
The Dickkopf1 (DKK1) gene encodes a protein mainly involved in Wnt and other signaling pathways.
Inhibition of DKK1 in Hep-2 cells reduces their proliferation, colony formation, cell migration, and invasion in vitro\cite{Shi2014}.
Pang et al.\cite{Pang2018} has proved that upregulation of DKK1 in SBC-3 cells (human small cell lung cancer) enhances their proliferation, colony formation, cell migration, and invasion in vitro, as well as bone metastasis in vivo. 
Increased DKK1 levels in HNSCC tissues is correlated with elevated VEGF-C and beta-catenin\cite{Shi2014}.
DKK1 expression is significantly associated with smoking, alcohol abuse, human papillomavirus status\cite{Chakraborty2020}, tumor site, tumor invasion, and pathologic stage in HNSCC patients.\cite{Gao2018}.
The mRNA expression of DKK1 and DKK3 is elevated in human papillomavirus (HPV)-negative HNSCC\cite{Hu2020}. Overexpression of DKK1 indicates adverse OS in bladder urothelial carcinoma (BLCA)\cite{Wei2020}, HNSCC\cite{Chakraborty2020}\cite{Hu2020}\cite{Wei2020}, and pancreatic adenocarcinoma (PAAD)\cite{Wei2020}. %Moreover, DKK1 is increased in HPV\+ HNSCC, leading to the worst prognosis of the patients. 
%\cite{Shi2014}\cite{Gao2018}\cite{Chakraborty2020}(non-TCGA)\cite{Hu2020}\cite{Wei2020}
%Wei et al\cite{Wei2020} conducted the analysis of prognostic value of DKK1 expression in human cancers based on bioinformatics tools, including UALCAN, GEPIA2 (dataset from TCGA)\cite{Tang2019}, and DriverDBv3 databases.
%using http://gepia2.cancer-pku.cn/detail.php?gene=DKK1 => \cite{Tang2019}
%GEPIA is a commonly used interactive website that plots expression profiles of given genes (from TCGA database)


%STC2:
The miR-381 suppresses cell proliferation, migration, and invasion in the HNSCC SCC-4 cell line through targeting stanniocalcin 2 (STC2) and participates in HNSCC development probably via the FAK/PI3K/Akt/mTOR signaling pathway\cite{Ma2020}.

% PGK1
Phosphoglycerate kinase 1 (PGK1 or PGK-1), a glycolysis enzyme, is responsive in cisplatin-resistant HNSCC cell line (H-1R). 
The resistance is associated with the up-regulated expression of PGK1, CD55, ATP-binding cassette transporter genes (e.g., MDR1, MRP1, and MRP2), and down-regulated expression Caveolin 1\cite{Nakamura2005}.

%SURF4
The patients with tumors (such as glioma, breast cancer, lymphoma, pancreatic cancer, adrenocortical carcinoma, and sarcoma%; not HNSCC
) exhibiting high Surfeit gene 4 (SURF4) expression have significantly shorter overall survival than low SURF4 expression.
SURF4 can induce cellular transformation and cell migration in vitro and has increased tumor growth ability in vivo\cite{Kim2018a}.
%NIH/3T3 cells (Mus musculus, mouse), 293T cell
%Loss of contact inhibition leads to phenotypic changes and promotes foci formation in vitro
%http://www.canevolve.org/AnalysisResults/AnalysisResults.html
%web-based tools survival (not TCGA?)

% USP10
Ubiquitin specific peptidase 10 (USP10), one of the autophagy-related genes (ARGs), may serve as a prognostic biomarker and target for various human cancers, including epithelial ovarian cancer, colorectal cancer, and HNSCC\cite{Ren2020}.
Functional annotation - by gene set variation analysis (GSVA) and gene set enrichment analysis (GSEA) - reveals that USP10 is significantly enriched in many critical pathways correlated with tumorigenesis of HNSCC, including the p53 pathway, IL-2–STAT5 signaling, transforming growth factor beta (TGFB or $TGF\beta$) signaling, and PI3K/Akt/mTOR signaling\cite{Ren2020}.
% 13 ARGs (GABARAPL1, ITGA3, USP10, ST13, MAPK9, PRKN, FADD, IKBKB, ITPR1, TP73, MAP2K7, CDKN2A, and EEF2K) with prognostic value were identified in HNSCC patients
%found at Human Protein Atlas dataset (HPA), TCGA, and (GSE6631): A total of 44 patients (22 HNSCC samples and 22 normal samples) were obtained from GSE6631. https://www.proteinatlas.org/ENSG00000103194-USP10/pathology/head+and+neck+cancer

%FOXA2
Prognostic signature integrated of DNA methylation, gene expression, and clinical information could provide a prognostic prediction value for HNSCC patients. Forkhead Box A2 (FOXA2) is significantly associated with the poor prognosis of HNSCC through the studies of CpG-based DNA methylation and RNA expression approach\cite{Shen2017a}.
%32 HNSCC patients had both tumor and adjacent non-tumor tissue samples in TCGA cohort, which was used as the discovery set to identify differential methylation CpG sites.
%  + 0.0256 × cg03774514 FOXA2 => "+" coefficients; higher prognostic score was significantly associated with shorter survival in the training set; 
% https://www.ncbi.nlm.nih.gov/research/pubtator/?view=docsum&query=28852427
%A multi-stage screening strategy => univariate Cox regression was used to evaluate their association with overall survival in the training set, which identified 15 CpG sites with P < 0.05. Further, SIS analysis was performed to further screen out a stable probe combination. Seven of the 15 candidate CpGs were identified, including cg13495205, cg07110405, cg03774514, cg09137696, cg19655456, cg03146625, and cg21546671 (Fig. 2c, Additional file 1: Table S1), mapped to AJAP1, SHANK2, FOXA2, MT1A, ZNF570, HOXC4, and HOXB4, respectively. Using coefficients generated from Cox model, we calculated a prognostic score for each patient based on individualized values of the seven genes (Fig. 2d): prognostic score, methylation = 0.0054 × cg13495205 AJAP1  + 0.0318 × cg07110405 SHANK2 + 0.0256 × cg03774514 FOXA2  + 0.0063 × cg09137696 MT1A  + 0.0013 × cg19655456 ZNF570 − 0.0297 × cg03146625 HOXC4  − 0.0157 × cg21546671 HOXB4 .
%Association between gene expression and methylation. Left panels show correlation of a AJAP1, b SHANK2, c FOXA2, d MT1A, e ZNF570, f HOXC4, and g HOXB4 expression (X-axis) with methylation (Y-axis). Right panels show Kaplan-Meier survival plots of gene expression from the TCGA cohort. 


% STIP1; non-TCGA
Stress-induced phosphoprotein 1 (STIP1, also known as HOP, P60, STI1), is increased in the poor survival patients of ovarian cancer\cite{Chao2013}\cite{Cho2014}. 
The auto-antibody against STIP1 in serum could be a useful diagnostic biomarker for early-stage esophageal squamous cell carcinoma\cite{Xu2017}.
%STIP1 could bind to bind Hsp70 and Hsp90.


%IL-19 specifically activated an intracellular signal and induced proliferation of the cells, which indicated that IL-19 may act in an autocrine manner in oral cancer.
%IL19 in esophageal squamous cell carcinoma\cite{Hsing2013}, the effects of IL-19 on the esophageal SCC in vivo and in vitro. CE81T cells
%IL-19 may be involved in the pathogenesis of systemic inflammatory diseases. IL-19 is also involved in various inflammatory diseases such as psoriasis, asthma, and rheumatoid arthritis.
% the Chi-Mei Medical Center Institutional Review Board (IRB9705-003). n=60


%%%%%% good guy
%ZNF557 is a tumor suppressor in HPV-positive HNSCC???
%Oncogenic human herpesviruses (e.x. EBV and KSHV) 
%Oncogenic human viruses are silenced through the activities of two members of the Kruppel-associated box (KRAB) domain-zinc finger protein (ZFP) (KRAB-ZFP) epigenetic silencing family, revealing a novel STAT3-KRAB-ZFP axis of virus latency.
%Epstein-Barr virus (EBV) and Kaposi’s sarcoma-associated herpesvirus (KSHV) are silenced by SZF1 and ZNF557, two members of the KRAB-ZFP repressor family\cite{Li2018c}


%ZNF266, % or ZNF16, HZF1?
%MYO1H associated mandibular prognathism\cite{Sun2018}

% FCGBP
% IgG Fc binding protein ($Fc\gamma BP$), (Fc Fragment Of IgG Binding Protein) 
Fc fragment of IgG binding protein ($Fc\gamma BP$ or FCGBP) is expressed in normal thyroid and, more significantly, it is down-regulated in papillary and follicular thyroid carcinomas\cite{ODonovan2002}\cite{Griffith2006}.
%in thyroid biopsies might help to make the difficult distinction between a thyroid follicular adenoma and a follicular carcinoma.\cite{ODonovan2002}\cite{Griffith2006}, 
%gene ID @gene@8857

\acrfull{LOC148709} has a contribution to the Warburg effect on esophagus cancer\cite{Liu2019}.
In our current study, LOC148709 is also suggested as a biomarker of \acrshort{hnscc} (see Table \ref{table:table3} on page \pageref{table:table3}).
 

%EVPLL in cDNA project

% PNMA5
Paraneoplastic Ma 5 (PNMA5) promotes apoptosis signaling in HeLa (cervical cancer) and MCF-7 (breast cancer) cells\cite{Lee2016}.

%The endogenous ligand of G protein-coupled receptors (GPR7) is IQCN (previous/former name as KIAA1683), and NPB (Neuropeptide B)\cite{Andreis2005}. GPR7 is associated with a poor prognosis of prostate cancer\cite{Cottrell2007}. 

%calcium/calmodulin-dependent protein kinase II inhibitor (CAMK2N1) is the endogenous inhibitor of CAMK2, which is activated by an increased intracellular Ca2+ concentration.SYT12 plays a critical role in oral cancer and may be a novel therapeutic target. (PMID31598163)
%Overall survival times (as determined by Kaplan-Meier analysis) were markedly longer in patients that had elevated CAMK2N1 expression compared with patients with negative or low CAMK2N1 expression. The miR-182-5p plays a vital role in controlling OSCC cell apoptosis and proliferation by regulating CAMK2N1 expression, which was found to be reduced in OSCC. Down-regulation of miR-182-5p expression attenuated tumor growth, cell survival, and proliferation in vivo through the regulation of the AKT/ERK1/2/NF-κB signaling pathways.
%CAMK2N1 operated as a tumor suppressor gene in patients with HNSCC.
%: 26 results of cancer: ; 1 HNSCC cell line (CAL-27) all said "good guy"
%\cite{Li2018b}
%EMT genes (TCF3, CAMK2N1, EGFR, and IGFBP4) 
%x c) CCLE_RNAseq_rsem_genes_tpm_20180929.txt.gz

%NDFIP1:
%The adaptor protein Nedd4-family interacting protein 1 (NDFIP1), was reported as candidate biomarker for breast cancer\cite{Tian2020}. was both better prognosis in the overall survival (OS) and relapse free survival (RFS)
%MCF7 cells 
%Howitt et al showed that NDFIP1 knockdown results in the loss of phosphatase and tensin homolog nuclear compartmentalization, promotes cell proliferation, and alters the cell cycle
%NDFIP1 is a direct target of miR-155; MiR-155 promotes uveal melanoma cell proliferation and invasion by regulating NDFIP1 expression.
%=> good guy: Zhang et al\cite{Zhang2019a} report that miR-873 promotes Warburg effect in HCC cells by increasing glucose uptake, extracellular acidification rate (ECAR), lactate production, and ATP generation, and decreasing oxygen consumption rate (OCR) in HCC cells. Mechanistically, we show that miR-873 activates the key glycolytic proteins AKT/mTOR via targeting NDFIP1 which triggers metabolic shift. We further demonstrate that enhanced glycolysis is essential for the role of miR-873 to drive HCC progression. hepatocellular carcinoma\cite{Zhang2019a}
% MiR-873 promotes the proliferation, migration, and invasion of liver cancer cells via NDFIP1 in HEK293T cell.
%  the adaptor protein Nedd4-family interacting protein 1 (NDFIP1) plays a key role in the ubiquitination and nuclear translocation of PTEN. It represses cell proliferation of melanoma and thus acts as a tumor suppressor.(PMID: 29333944)

%DKC1 (dyskeratin) is related with HNSCC\cite{Smith2010}.
%DKC1 is the tumor suppressor gene in HNSCC
% Fanconi anemia (FA) and dyskeratosis congenita (DC) are rare inherited syndromes that cause head and neck squamous cell cancer (HNSCC).






\subsection{Limitations of the work}

Although we combine the power of genome-wide scanning and an optimal cutoff finder for survival analysis, the study has some limitations.
% first limitation:
We are aware of the importance of direct assessment of protein products comprising the basic functional units in cancer cells' biological processes. The announcement of the Cancer Proteome Atlas (\acrshort{tcpa}, http://tcpaportal.org) excites the cancer research community\cite{Li2013c}\cite{Li2017b}. By the utility of the \acrfull{rppa} or \acrfull{rpma}, a microarray kind of "Western blots" in the \acrshort{tcpa} could help to test our hypotheses from \acrshort{rnaseq} study.
However, in the \acrshort{tcpa} database (v3.0\cite{Chen2019}), there are only 237 antibodies available, not covering our candidates so far. %  USP-10 in HPA
%> semi_join(pvalueTex_bad, HPA_bad, by=c("V1"="Gene"))
%validation at protein level:
%At the same time, we found our candidates (DKK1, CAMK2N1, STC2, PGK1, SURF4, NDFIP1, STIP1, DKC1) are also on the list of unfavorable prognostic genes for HNSCC from \acrfull{hpa} (available at https://www.proteinatlas.org/humanproteomepathology/head+and+neck+cancer, accessed 28 September 2020). The ZNF557, ZNF266, and FCGBP are on the list of favorable prognostic genes as well.
% > semi_join(pvalueTex_good, HPA_good, by=c("V1"="Gene"))
%There are 628 genes that consensus of pvalue005KM (6429 genes) and HPA (bad+good in HNSCC)
%%%%%%%%%% modified discussion
%\begin{MyColorPar}{red}



% second limitation:
%\subsubsection{Cancer type-agnostic study}
Our strategy still has the strength to explore the more possible biomarkers from \acrshort{rnaseq} datasets in cancer research.
% *** yes here
In our previous work, altered glucose metabolism (e.k.a. the Warburg effect\cite{Warburg1956}) promotes the progression of \acrshort{hnscc}, which is partially attributed to the \acrlong{slc2a4} (\acrshort{slc2a4}, or \acrlong{glut4}, \acrshort{glut4}) and \acrfull{trim24} pathway\cite{Chang2017b}\cite{Mani2020}.
% Cited In for PMID: 28061796
Lactic acidosis induced \acrshort{glut4} overexpression was also found in lung cancer cells\cite{Prado-Garcia2020}. 
%The overexpression of \acrshort{glut4} will lead to higher glucose uptake in proliferating cancer cells\cite{Mani2020}.
%There is a trend toward cancer type-agnostic study. 
Currently, Pembrolizumab and nivolumab's success has been based on a common biomarker (e.g., \acrshort{pd1}) crossing several types of cancer.
It shows a model of tumor type-agnostic therapy\cite{Yan2018}.
%*** PD-L1/PD-L2 is a biomarker; Keynote-012
There are several common biomarkers of \acrfull{ici} under evaluation, including \acrfull{til}, \acrfull{ifng}, and \acrfull{tmb}\cite{Gavrielatou2020}.
The other \acrshort{ici}, anti-\acrshort{lag3} (pelatlimab), is currently evaluated under the phase I/IIA\cite{Cristina2019} (ClinicalTrials.gov Identifier: NCT01968109) and II-IVA\cite{Neal2019} (NCT04080804) studies.

In line with tumor-agnostic research, we plan to explore common biomarkers crossing TCGA diseases. However, the \acrshort{gdc} provided standardized data frames that could not directly fit our workflow's scope. Before the global genes scanning process, it is necessary to re-format, transpose and, merge the 528 patients' clinical datasets and correlated 20,500 expressions of bio-specimen. This process should be carefully curated to confirm the data integrity within the correct definition\cite{Rendleman2019}. We also plan to upgrade our R script of the cutoff engine to the C++ language and source it in the Rstudio server. The high performance of C++ could speed up the critical steps in this workflow involving heavy computation of matrix data\cite{Woodward2020}. Moreover, it is possible to introduce the Rstudio Shiny app (https://shiny.rstudio.com) as a web-based tool (named "pvalueTex" ) packaged with our workflow in the future.




\subsection{Future research directions}
\subsubsection{Laboratory validation}
It is encouraged for multidisciplinary studies that use complementary computational and experimental approaches to address challenging cancer research.
%下一篇文章的內容。目前在計畫中,但還沒有開始進行
These in vitro and in vivo validation experiments will be undertaken in our laboratory.
We plan to analyze mRNA (e.x. qRT-PCR) and protein (e.x. Western blot) of HNSCC cell lysate to confirm the candidate genes' expression.
%mRNA expression level in HNSCC cell lines (其他文章已有發表 overexpression)
The effect of overexpression and knockdown of the gene by lentiviral clones should be observed on cell function assay (e.x. proliferation, migration, and invasion) and mouse xenograft model (e.x. tumor growth). %這些都需要一年以上的時間
%their rationale for looking into TMSB4X is not strongly supported by the available data.

% 其實文章真正的目標是  bioinformatics workflow,而不是 discovered biomarkers itself (see TCGA marker paper\cite{Lawrence2015a})
Moreover, this bioinformatics paper provides targets and supports the community's rationale for looking into these HNSCC candidates by in vitro and in vivo validation. %Collectively, these findings provide new insights into HNSCC and suggest that shared and unique alterations might be leveraged to accelerate progress in prevention and therapy across tumour types\cite{Lawrence2015a}.
%We open invitation for research community by potential support their rationale for looking into these HNSCC candidates.
%邀請科學家們一塊兒來繼續研究下去找出幫助癌症患者的方法
%the rest of the analysis seems fine, but 
%文章的初衷、跨領域:如同deep leaning Keras API,推廣給研究者一個有效的方法 
We aim to promote a reproducible bioinformatics\cite{Preeyanon2014}\cite{Kulkarni2018} workflow allowing successful repetition and extension of analyses based on the TCGA or other in-house HNSCC datasets. % to find more useful biomarkers. % or even other cancer types
% to allow successful repetition and extension of analyses based on original data
A good research reproducibility practice is necessary to allow the reuse of code and results for new projects.
%previously developed methodology to be effectively applied on new data, or to allow 
%reuse of code and results for new projects. %In other words, good habits of reproducibility
It may turn out to be a time-saver in the longer run.
When multiple scientists can reproduce a result, it will also validate our initial results and readiness to progress to the next research phase. 
%why reproducibility?
%Reproducible Bioinformatics Project provides a general schema and an infrastructure to distribute robust and reproducible workflows. Thus, it guarantees to final users the ability to repeat any analysis independently using UNIX-like architecture.
%Utility of this model would facilitate development of more individualized therapy for HNSCC patients and improve prognosis.
Once our laboratory or the community confirms those candidates as the targets, the compound screening\cite{Yang2004}\cite{Hsu2011}\cite{Pathak2021} could facilitate more individualized therapy for HNSCC patients.
%and improve prognosis.
% 1. Pathak, N. et al. Uncovering flexible active site conformations of SARS-COV-2 3Cl proteases through protease pharmacophore clusters and covid-19 drug repurposing. ACS Nano 15, 857–872 (2021). 陽明交大


\subsubsection{TCIA validation}
https://wiki.cancerimagingarchive.net/display/Public/TCGA-HNSC
https://cancer.digitalslidearchive.org/#!/CDSA/hnsc/TCGA-BA-4076/TCGA-BA-4076-01A-01-BS1?slide=TCGA-BA-4076-01A-01-BS1.4aef481c-f634-4ffc-9fb3-609f59a8caf9.svs
TCGA Pathology Images => The Cancer Imaging Archive (TCIA) by Deep learning (R/Keras framework)
%數位模擬,能讓臨床醫師和研究人員更快速了解細胞成長後的狀況,而不是花費時間等待細胞的培養。
image-based phenotypes and their underlying genomic data 
https://www.biorxiv.org/content/10.1101/2020.03.10.985887v1.full
https://github.com/Serian1992/ImgBio.

%The Stanford Tissue Microarray Database (TMAD; x http://tma.stanford.edu) 
%https://tma.im/cgi-bin/ontologyBrowser.pl?anchor=C34447; there is few HNSCC slides

*** interpretable or explainable CNN

https://www.cancer.gov/research/areas/diagnosis/artificial-intelligence
% NCI
Aiding the Genomic Characterization of Tumors: 
deep learning could predict commonly mutated genes and mRNA expression from the images
*** A deep learning model to predict RNA-Seq expression of tumours from whole slide images\cite{Schmauch2020}
Preprocessing of whole-slide images. The application of deep-learning algo- rithms to histological data is a challenging problem, particularly due to the high dimensionality of the data (up to 100,000 × 100,000 pixels for a single whole-slide image) and the small size of available datasets. We divided the whole-slide images into squares of 112 × 112 μm (224 × 224 pixels) called “tiles”, and used the Otsu algorithm62 (as implemented in python package skimage) to select only those containing tissue, excluding the white background. We sampled a maximum of 8000 such tiles from each slide. We then extracted 2048 features from those tiles with a 50-layer ResNet63 pretrained on the ImageNet dataset64 (using the Keras implementation), such that a slide could be represented as a 8000 × 2048 matrix.
For the first phase of this work (transcriptome prediction), we accelerated the training of our models through a simple preprocessing step inspired by simple linear iterative clustering (SLIC)65: we used the k-means algorithm (as implemented in python package libkmcuda) to create 100 clusters (supertiles) of tiles on the basis of tile location on the slide, and we averaged the features of the tiles within each cluster. The use of these supertiles reduces the dimensions of a slide to 100 × 2048. The model was first trained on this reduced dataset, with all the TCGA data. Then, for specific organs, fine-tuning was achieved with full-scale data from the organs concerned only.

%AI methods can also be used to identify specific gene mutations from tumor pathology images instead of using traditional genomic sequencing. For instance, NCI-funded researchers at New York University used deep learning (DL) to analyze pathology images of lung tumors obtained from The Cancer Genome Atlas. Not only could the DL method accurately distinguish between two of the most common lung cancer subtypes, adenocarcinoma and squamous cell carcinoma, it could predict commonly mutated genes from the images.

Understanding the Method Behind the Machine
interpredability

%One challenge of AI, and DL specifically, is the “black box” problem: not fully understanding what features of the data a computer has used in its decision-making process. For example, a DL algorithm that predicts the optimal treatment for a patient does not provide the reasoning it used to make that prediction. Additional efforts are needed to reveal how algorithms arrive at a decision or prediction so that the process becomes transparent to scientists and clinicians. Making these algorithms transparent could help researchers identify new biological features relevant to disease diagnosis or treatment.
***  unconscious biases:
Incorporating information about biological processes into the algorithm is likely to improve its accuracy and decrease dependence on large amounts of annotated data, which may not be available. One danger of the “black box” problem is that DL may inadvertently perpetuate existing unconscious biases. Researchers need to carefully consider how potential biases affect the data being used to develop a model, adopt practices to address and monitor those biases, and monitor performance and applicability of AI models.

%With increased investments, NCI’s efforts to realize AI’s potential will lead to more accurate and rapid diagnoses, improved clinical decision-making, and, ultimately, better health outcomes for patients with cancer and those at risk. Updated: August 31, 2020
%
TMA by deep learning (CNN) mRNA expression associated with pathological images
*** TCGA TCIA Pathology Images with deep learning: gene expression on pathological section, or digital tissue microarray, TMA
recurrent neural network for NLP

%佩戴可監測心率、體溫、血壓及其他關鍵數據的裝置,醫師也因此能夠獲得更多資料以更深入瞭解患者的病情,不再依靠病人的回憶進行判斷,診斷的準確性也會隨之提升。

%在生物醫學領域,氨基酸或DNA等序列的使用可說是相當普遍,由於序列可視為一種具有隱性結構的語言,因此自然語言處理模型所採用的架構便可運用在理解並生成生物醫學領域的序列。

%the Rstudio Shiny app (https://shiny.rstudio.com) as a web-based tool (named "pvalueTex" ) packaged with our workflow in the future.
%with more TCGA features (holistic)


\subsubsection{Holistic cancer care} 
% carcinogenesis by epigenetic control (placebo/nocebo), Bruce Lipton introduced mind-brain-body system. (gene expression is not an independent $X_mrna$, which is correlated with other $X_body, X_stress, X_inflammation, X_ros$)
% cancer care with person-central approach, PCA; Cullen, Carl Rogers

%TCGA treated by surgery and more; TNM available yes

%discussion: the third limitation
Even there are eighty one ($X_1 ... X_81$) physical, pathological and social features of patients available for survival modelling in the TCGA, 
% since it has comprehensive clinical features, 
such as age, gender, residual tumor, vital status, days\_to\_last\_followup, cancer stage, smoking duration, exposure to alcohol, asbestos, radioactive radon, the TCGA did not collect other holistic features from their participants.
% Input $X_1...X_n$ (e.x. patients features: such as age, gender, gene expression, cancer stage, 
Going for holistic cancer care\cite{Mehta2019}\cite{Iftikhar2021} spiritual and emotional condition is equal important comparing with physical and social status.
%
World Health Organisation has called worldwide for delivering people-centred (or person-centred) care from the physical, emotional, socioeconomic and spiritual perspectives\cite{WHO2015}\cite{Ling-ChengMong2021}.
Core principles can be taken to achieve people-centred and integrated health service delivery. 
First principle is holistic – focusing on physical, socioeconomic, mental and emotional well-being. 
Holistic care: care and see the person as a whole with psychological, social and environmental levels of needs and goals.
%that considers  factors rather than just the symptoms of disease or ill-health.
Second principle is Empowering – supporting people to manage and take responsibility for their own health. 
Empowerment: the process of supporting people and communities to take control of their own health needs resulting, for example, in the uptake of healthier behaviours or the ability to self-manage illnesses.
People-centred care requires that people have the education and support they need to make decisions and participate in their own care.
%Glossary:
%People-centred care: an approach to care that consciously adopts individuals’, carers’, families’ and communities’ perspectives as participants in, and beneficiaries of, trusted health systems that respond to their needs and preferences in humane and holistic ways. People-centred care also requires that people have the education and support they need to make decisions and participate in their own care. It is organized around the health needs and expectations of people rather than diseases (7,8) 
%Person-centred care: care approaches and practices that see the person as a whole with many levels of needs and goals, with these needs coming from their own personal social determinants of health.
%(1. World Health Organization. WHO global strategy on people-centred and integrated health services: interim report. https://apps.who.int/iris/handle/10665/155002 (2015).)

The spiritual care is the key of holistic healthcare in dental practice\cite{Ling-ChengMong2021} or even for cancer patient. Being mindful to patient’s mental health and emotional changes during the dental care. Taking patient’s socioeconomic statement and supporting system thoroughly for building a holistic consideration. Learning to listen proactively.
Being aware of patient’s spiritual needs during/after treatment.
%Applying the holistic healthcare strategies, such as the shared decision making, to collect more spiritual-related information of patients in the treating process and adjusting the approaches dynamically.\cite{Ling-ChengMong2021}

%% PCA and ECA
Therapeutic relationship is an essence towards spiritual care.
Carl Rogers developed person-centered theory that the therapist is providing the appropriate social-environmental conditions for their patients\cite{Rogers1957}\cite{Joseph2004}\cite{Griffiths2013}.
Person-centred theory holds that the innate tendency towards growth and development is facilitated by the six conditions described by Rogers in 1957\cite{Rogers1957}.
The research implication is to investigate the relationship between the patient’s intrinsic motivation to explore his or her trauma-related thoughts and feelings and the nature of the therapeutic relationship as defined by the six conditions\cite{Rogers1957}.
Dealing with negative emotions. As the patient feels intense negative emotions of fear, despair, sadness and so on, it is not the patient-centred therapist’s task to try and remove these feelings: The task is simply to stay with the patient’s experiences. 
%The research implication is to investigate the necessity and sufficiency of the six conditions.
%Rogers, C. R. A Theory of Therapy, Personality, and Interpersonal Relationships: As Developed in the Client-centered Framework. in Psychology: A Study of a Science. Study 1, Volume 3: Formulations of the Person and the Social Context (ed. Koch, S.) 184–256 (McGraw Hill, 1959).\cite{Rogers1959}


%PCA理論
治療關係的核心條件: 說好話 眼對眼 engaging 用真心
以心印心 engaging
身心相連 body-mind connection
捨末求本
身心靈全人醫療,展現魅力、慈悲心,靈性成熟的醫師,用藥物或手術來治療病患時,
「他自己本身」就具有療癒效果
「把自己當藥方開出去」

empowering
(一)雙方有心理的接觸。
(二)當事人正感到焦慮或不一致(incongruence)。 
(三)治療者在治療關係中是一致的、真實的或真誠的。
(四)治療者能夠對當事人表現出無條件的正向關懷。
(五)治療者能夠同理的瞭解當事人的內在,並努力傳達「我懂你的」。
Personhood: Carl Rogers
尋找生命的意義
真誠、能深刻地善解人意
非認知性核心能力
正念止觀(在當下)
人人有自癒能力
真誠 (Genuineness/Validity/Bona Fides)
內在聲音與外在態度一致/明心見性/超越的意識狀態
(Congruence/Consistency)
共情理解(設身處地/同理心/除我執)
(Empathic Understanding of Reference/Track)
無條件關懷(不批判/頓悟)
(unconditional positive regard)
***引導出案主的無條件自我關懷 => 療癒/復原力(resilience)
諮商員的
靈性功力
治療關係: patient said "I would like to talk about myself more when I visit this doctor who I trusted" % 不自覺就講出,不自主就想要聽從他的建議,不再有抗拒
I find that when I am closest to my inner, intuitive self, when I am somehow in touch with the unknown in me, when perhaps I am in a slightly altered state of consciousness, then whatever I do seems to be full of healing. Then simply my presence is releasing and helpful\cite{Rogers1979}.
%當最接近我的內在、直覺的自我,當我與我內在那不知名的角落(the unknown in me)接觸時(靈性),當我有一些不一樣的意識狀態時(明心見性),我發現,單純只是我的存在(presence)就能鬆綁案主而且具有療效的。
%Rogers, C. R. The Foundations of the Person-Centered Approach. Education 100, 98–107 (1979).
「診後說三句,更勝良帖十方」


%% why holistic?
中醫人體觀-五臟與經絡情緒 工作坊《課程簡介》情緒是人體自身的氣,我們因情緒而調動它,也因它內傷,誘發情緒的原因有外在和內在,外在多跟經絡接收訊息的能力有關,當外在的影響如果沒有適當阻斷的時候,也會影響到五臟功能;內生的情緒則多以活出自我價值感有關,當我們活出此生想要體驗的人生與成就,我們就能夠不受情緒控制而保持內在平靜,這一切的源頭皆與我們是否活出真實的自己有關。
養生和治療的原則:
一、某些病症可以透過食療而改善的會建議自行調養
二、某些病症已需要醫療輔助的時候會建議醫食同時進行勿偏癈其一
三、某些病症需要醫療系統監控的時候會建議多找幾家醫院醫師綜合性評估
四、沒有一勞永逸的食療,不宜執著特定的食療神話
五、體質會變,所以變的是療法,而不是讓體質去適應固定療程
六、認識自己身體的狀態比吸收營養資訊更重要
七、沒有好或不好的飲食或療法,只有適不適合自身體質的療法
情志養生 悲喜憂思苦 七情
人俯仰於天地之間
順從四季氣候變化
保養正氣陶冶性情
自我療癒身心靈疾病
%自他不二 心存正念 向內看,解決苦之源
%1.衷心懺悔
%2.真心感謝
%3.誠意祝福
%4.永存善念 慈悲
%5.心無恐懼: 情志養生
每位醫師都可以成為「創傷知情者」
幫助我們身邊的病患, 懂他的心靈創傷
讓他有安全感, 才有機會改變疾病的走向

“每一位患者都有自癒能力, 我們知情之後, 也要逐步讓他本人知情, 看見之後, 在良好的「治療關係」中, 協助他們漸漸找回自己的療癒。(以他們自己的腳步)”
解說:
患者的自癒力就是復原力(resilience),  強調「治療關係」(安全、 信任、 分享權力、 自決) 以及”知情”(暸解過往創傷經驗對自身的影響,進而開始療癒的過程)的重要性。

%% why holistic?
Bruce Lipton introduced mind-brain-body system.
Lipton 1992, epinephrine and histamine signals in endothelium, %心智作用
 In a state of fear, the brain releases chemicals associated with fear, which are stress hormones and inflammatory agents.
dopamine, oxytocin, vasopressin, and growth hormone will be released by central nerve system in response to the perception of love.

Epigenetics control says that genetic expression is directly due to the environment and our perception of the environment\cite{Gustafson2017}.
The microenvironment of HNSCC requires further exploration and understanding using a multiparametric approach.
Bruce Lipton: the biology of belief %信念的力量
placebo effect and the old belief of genetic determinism\cite{Mokhtari2011}\cite{Kobisi2012}\cite{Gustafson2017}
%biowell 勿膨脹自我: 我非常擔心,所謂的科學實驗,研究的疾病,發明的藥物,都會潛藏一種危機,因為它的療效與科學證據,都是建立在「意識統一場域」中,科學家在研究時,一定希望有效,能夠發現別人沒有發現且有用的事情,所以「成功」還是會帶有「人為的意念」。
%那麼,這個結果不「公正」。甚至,當原來藥物發明者的心念改變了,(開始只想賺錢),那麼藥物的作用就更無法掌握。也許,藥物副作用會很嚴重。而目前科學的解釋是,只是因為個體的基因差異。

%% 1. Kobisi, W. Mind inhibitors. in The 30th International Congress Of The World Federation Of Hemophilia, July 8‐12 vol. 18 170–171 (John Wiley & Sons, Ltd, 2012).
As a person with hemophilia (PWH), my passion led me to conduct personal research to uncover the main cause of my drawbacks in my education, my career, and my romantic life. I have experienced unjustified conflicts in my social relations, feeling stale in mind and body, indefinite self-image, emotional instability, and successive spontaneous bleeds despite the continuous factor replacement. My research is based on my experience of living with hemophilia for 37 years, reading countless books on personal development and the mind-body relationship, and having profound discussions about these experiences with my wife. Findings: Through my research, I uncovered that I used to suffer from what I call 'mind inhibitors' these were the hidden cause of my life disorders and spontaneous bleeds. These mind inhibitors are responsible for locking the body factor receptors, blocking body energy centres (like the solar plexus), discouraging the super-intelligent body cells from running their auto-healing system, and paralyzing the instinctive high-speed cellular response to external molecular treatment. Mind inhibitors originate from different sources like thinking of oneself in a victim role, anxiety about medical treatments, lack of financial stability, negative attitudes, self-rejection, and contradictory thoughts and feelings. Conclusion: Mind inhibitors should be handled as seriously as biological inhibitors. As Dr. Bruce Lipton explains in his book, 'The Biology of Belief': DNA does not control our biology; instead DNA is controlled by signals from outside the cell. Our bodies can be changed as we retrain our thinking, since thoughts are made of energy that affects the cellular energy level, and thus everything is created twice: once in our minds and then the physical creation follows. New holistic approaches should be considered for treating PWH as a complete synchronized system of Soul, Mind, and Body; treatment should include more than just factor adjustment. Factor adjustment only treats symptoms, but a holistic approach would also address the patient's need for a subconscious paradigm shift in the person's thought and belief system.\cite{Kobisi2012}

%%


genetic determinism 所以我們的科學家依舊在研究,基因對癌症的影響(單向)
The genetics of the cell give us all kinds of potentials. % 自性空

%冤親債主的意念 unknown X factors
[Results] In summary, those 10 candidate biomarkers, clinical T stage, and surgical margin are independent prognosis factors in \acrshort{hnscc}.
Thus, the prognosis model with coefficients is established from \acrshort{tcga} \acrshort{hnscc} cohort. The important input $X_1...X_n$ should be patients' features: age, specific 20 gene expressions, clinical T stage, and surgical margin.
依舊是以管窺天
% https://link.springer.com/referenceworkentry/10.1007%2F978-1-4419-1428-6_575
%癌症與身心靈的重要關係:中醫、citation? emotional centered approach




% so do the cancer
Individual treatment and avoiding assumptions. Although it is useful for therapists to have a knowledge of the trauma literature on the general reactions of people who experience trauma, the therapist should not assume that this knowledge necessarily applies to each and every client\cite{Joseph2004}.

Cullen's lecture notes reveal a sophisticated clinical understanding of mind-body interaction,13,16 centred on a physician's therapeutic intentions and his attunement to patient sympathy.
we educate them
%1. Mong, L.-C., Liao, F., Chiou, J. & Chiang, P. Tips for Integrating Spiritual Care Delivery in Dental Education. in Asia Pacific Medical Education Conference. Short Communication 5 (National University of Singapore, 2021).\cite{Ling-ChengMong2021}



%騙過你的病-安慰劑效應 
The Placebo Effect (Emmanuelle Sapin, Pascal Goblot)
%1. Sapin, E. & Goblot, P. L’effet placebo (English: The Placebo Effect) [enregistrement vidéo] : le formidable pouvoir de l’auto-guérison. 53 PP-Quebec (Cinéfête, 2014).\cite{Sapin2014}
% https://biblio.bdeb.qc.ca/in/sim/faces/details.xhtml;jsessionid=E3DD39109012D0BEA598E6E5B66A4FD2?id=p::usmarcdef_0000058243&posInPage=8
1. Finniss, D. G. Placebo Effects: Historical and Modern Evaluation. in International review of neurobiology vol. 139 1–27 (2018).\cite{Finniss2018}
%蓋倫劑量 placebo effect 身心靈醫學的核心
In 1772, Willian Cullen, a British physician, wrote that  "I own that I did not trust much to it, but I gave it because it is necessary to give a medicine, and as what I call a placebo. If I had thought of any internal medicine it would have been a dose of the Dover's powders."\cite{Kerr2008}\cite{Finniss2018}.
 (made up of Ipecac and opium) 
 Cullen studied the objective science of analytical chemistry. He also sense that the patient's subjective, psychosomatic or ‘sympathetic constitution’ determined his reaction to a prescribed therapeutic substance.
***Cullen developed a theory of "sympathy" (or empathy) which informed his ideas about clinical medicine dealing with psychosomatic theory of illness and mind-body therapeutics. % he do 炮製
Cullen's theory of sympathy is a mind-body function, a kind of ‘vital force’ that animated the human body, coordinated function, and transmitted sensation to target organs\cite{Forget2003}\cite{Kerr2008}.
This finely calibrated, rational use of active placebo treatment, in which the physician dispensed a weak, physiologically active substance in order to please and calm the patient rather than to cure the patient's underlying disorder, was a product of the rational scientific culture of the 18th century Scottish Enlightenment and its embrace of a grounded, empirical approach to medicine. %啟蒙運動

% how placebo effect works?
安慰劑效應的科學研究,驚人地發現現代西方醫學的不確定性。
前額葉背側區 (dorsolateral prefrontal cortex)  是形成安慰劑療效的關鍵(PCA「治療關係」的生理學依據)
the anterior cingulate cortex and the dorsolateral prefrontal cortex have been found to be activated by placebo administration.
\cite{Benedetti2005}\cite{Benedetti2011}\cite{Carlino2011}\cite{Bennett2018}
But modern research, especially the results of neuroimaging studies, has shown unequivocally that placebo-induced analgesia is real in the sense that the neural correlates of the sensation of pain are reduced, in the same way that they would be reduced by a decrease in the intensity of a pain-evoking stimulus or by the administration of an analgesic drug\cite{Eippert2009}\cite{Watson2009}. Today, pain researchers are united in the belief that placebo analgesia is real, in the sense that the neural processes that represent pain are reduced\cite{Bennett2018}.

%sympathy 富有同情心、慈悲心,靈性高度成熟的醫師,不是只用藥物或手術來治療病患,「他自己本身」就具有療癒效果。
The role of the physician was to be a gentle, sympathetic listener, who could interpret the patient's sympathetic disorder and select the proper remedy to treat it, whether by prescribing placebos to please (influence the emotional states of patients), or medicine to cure the patient's disorder.
placebo effect 安慰劑的研究,拓展了我們對醫病關係的理解,回歸到以病患為中心。
醫療處理的不只是藥物和手術,人更是必須處理的對象,把病患看作一個整體,要掌握身與心的個人特質。
安慰劑效應的研究,最後導向了良好醫病關係的思考,並成為醫生臨床醫術的元素之一:
「醫師開藥時,其實是把自己當藥方開出去」。



#正念 減壓,療癒疼痛\cite{Krummenacher2010}\cite{Bennett2018}、免疫系統、內分泌系統或自律神經系統相關的疾病,以及憂鬱症、焦慮症 
The idea of placebo as a diluted but active substance persisted into the 19th century17 but the term also came to mean a physiologically inert substance - bread or lactose, for example.18,19 Inert substances began to be used in comparative clinical trials during the first half of the 19th century,20 but this latter meaning of the term was given particular impetus at the end of the 19th century and the beginning of the 20th century, when inert substances were used as controls in pharmacological experiments.21 This shift in meaning may have reflected the rise of a new emphasis (or even metaphysics) in medicine, in which the molecular make up of an active drug assumed central importance, and any other positive benefits were seen as non-specific, placebo effects.\cite{Kerr2008}

prefrontal lobe 理智腦 意識 正念 杏仁核 prefrontal cortex runs conscious mind
「安慰劑」placebo effect,也已經有科學證明,與醫師、病患的意識(前額葉)功能,非常有關係。這也是負責同理心與愛的腦區。
物理作用、信號,雷射針灸
Therapeutic Relationship
Spirituality
慈悲心 compassion 
同情心 sympathy
同理心 empathy
non-verbal contact
前額葉內側 (medial prefrontal cortex)
額葉底部(orbitofrontal cortex)





*** make Figure 5 holistic:
meridian and emotion has mutual reaction
meridian effect tissue and organ
(autonomic nerve system, myofascial network, prefrontal cortex)
placebo-effect influence on drug discovery
博士班學生,越研究越發現,科學的不足,心理與靈性的力量,更是背後推動的冥冥力量,身心靈是合一的。#正念 除了可以減壓,還可以治癒疼痛




repentantly confess for not taking care of body and spirit, and sincere thanks for the hard work of my physical body

mindfulness and meditation for cancer
% 人的腦波一般分為β(12-30Hz)、α(8-12Hz)、θ(4-8Hz)、δ(1-4Hz),四種波。當我們處於清醒、警覺狀態時,腦波是處於高頻率的β波。隨著身心狀態逐漸放鬆,卸下防衛時,腦波頻率也會逐漸下降至α、θ波,此時身心狀態是處於最寧靜、創造能力最好、對事情洞察能力最佳的狀態。如果能夠安祥進入睡眠階段,則腦波會降至δ波。靜坐時不僅可以使腦波由快速的β波降至α、θ波
In human EEG studies, cortical theta wave (4–7 Hz) tends to appear during meditative, drowsy, hypnotic or sleeping states in older children and adults. Theta is a frequency lower than consciousness. Theta is actually a brain function associated with imagination.

Warburg effect and carcinogenesis
cancer therapy by 患者自癒
Mindfullness Meditation help 復原力(resilience)
Self-Compassion  慈悲
Self-compassion 
除我執
 global self-esteem:
自戀(narcissism)1. Neff, K. D. & Vonk, R. Self-compassion versus global self-esteem: Two different ways of relating to oneself. J. Pers. 77, 23–50 (2009).\cite{Neff2009}
%%%% healing and cure
Reprogramming cells:
Environmental signals controlling cell behavior and cell behavior include genetics. The environmental signal via signal transduction can go into the nucleus and selectively change the reading of our gene, epigenetics regulation.

the membrane proteins and the chromosomal proteins that control the DNA, called regulatory proteins. But even those proteins are controlled by environmental signals. It is not DNA to RNA to protein.

The new understanding is: environmental signals to regulatory protein to DNA to RNA and then to protein.

療癒 There are studies that showed the genetic readout of some inflammatory genes in a group of people who then went through a meditation process. After 8 hours of meditation, the activity of the genes changed. 
You can change your genetic activity by how you change your response to the environment. 

to recognize your perceptions, via signal transduction, are translated into biological behavior and your genetic activity.

Becoming aware of the subconscious source of our behavior gives us an opportunity to change our lives by rewriting the programs (in subconscious mind) of limitation or the things that interfere with us.

愛自己、蜜月;其實是 正念 mindfulness (因為自他不二)
You are creating a world of joy and love, and that’s called the honeymoon. When we fall in love, we stop focusing our conscious mind in thought and start keeping it present. 
It is called being mindful. 
You’ve been looking for yourself your whole life.
愛、性、正念止觀
Science has recognized that immediately after falling in love, we enter a period of mindfulness where we keep our conscious mind present.

You start running programs that are based on your conscious wishes and desires. All of a sudden, without the programs—without the subconscious programs—we begin to experience a heavenly life.
%%%%

%%%%%% carcinogenesis
mind-brain-body axis: subconscious fear -> inflammatory and stress hormones -> DNA/methylation mutation/mRNA expression alteration -> carcinogenesis of cancer.
95\% of cognitive behavior runs on subconscious program:

7\% 對於健康有不良影響 => 卻是惡劣競爭下的生存技能
Of the downloaded behaviors acquired before age 7, the vast majority—70\% or more—are programs of limitation, disempowerment, and self-sabotage. These programs were acquired from other people, not from ourselves. Again, being subconscious, these programs are occurring without conscious recognition and awareness. 
It effects on epigenetic regulation.

Less than 1\% of disease is associated with genetics. Over 90\% of disease is a total reflection of environment and especially our programming: the disempowering, self-sabotaging behaviors that we acquired in the first 7 years. 

Genes are not self-actualizing. The control of genes is not due to any inherent activity in the DNA itself. The change of genetic activity is due to the interaction of the cell with environmental signals.

the cell membrane through receptors picking up the signals and translating them into biology, which then sends signals into the nucleus, which then controls the genetic activity.

%%%%%

ground truth Y (e.x. patient's survival):\\[0.5cm]
%Y$:\\[0.3cm]
$Y = \beta_0 + \beta_1 X_1 + \beta_2 X_2 + \beta_3 X_3 + ... + \beta_n X_n + \epsilon$
gene expression is not an independent $X_mrna$, which is correlated with other $X_body, X_stress, X_inflammation, X_ros$
There is much work to be done in investigating the functional relationships more fully.\cite{Rogers1959}
We wish the TCGA does collect holistic features from their participants in the future. % electric healthcare record, EHR, enabled in holistic way
Deep Patient

We design a #holisticDental chart in our outpaitent department for Psycho-Social-Spiritual Evaluation\cite{Ling-ChengMong2021}.:
1. Emotional: stable/depression/ anxiety/ agitated/ perplexed
2. Interaction: expression disorder/ confusion/ hard to make decision/ uncooperative to dental treatment
3. Social: (disturbing living/ disturbing working/ disturbing other life aspect) by present illness
4. Function loss(subjectively): mastication/ pronunciation/ life quality/ state of mind
5. Self-expectation: lack of confidence/ low self-esteem/ social-withdrawal
6. Impact to life: financial problem/ family problem
7. Medical relationship: bad experience of dental treatment/ loss confidence to dentist/ anxious to treatment 
side-effects/ litigation experience
8. Other aspects
/no other specific issues

how to do: once holistic feature available, graph convolutional neural network (GCNN) is mandatory: 以免 genomics 光芒萬丈


Impact of stress on cancer metastasis\cite{Lutgendorf2010}\cite{Moreno-Smith2010}\cite{Du2020}\cite{Xu2021}
cancer initiation under stress\cite{Lutgendorf2010}\cite{Powell2013}
HNSCC\cite{Iftikhar2021}
Adverse psychosocial exposures in early life, namely experiences such as child maltreatment, caregiver stress or depression, and domestic or community violence, have been associated in epidemiological studies with increased lifetime risk of adverse outcomes, including diabetes, heart disease, cancers, and psychiatric illnesses. Additional work has shed light on the potential molecular mechanisms by which early adversity becomes “biologically embedded” in altered physiology across body systems.\cite{Berens2017}

We aim to promote a reproducible bioinformatics\cite{Preeyanon2014}\cite{Kulkarni2018} workflow allowing successful repetition and extension of analyses based on all the TCGA or other in-house cancer datasets.

Wish and bless the whole world
Dali Lama


%%%%%%%%%%%%%%%%%%%%%%%%%%%%%%%%%%%%%
\section{Materials and Methods} % 4

\subsection{Patient Cohort} 

The Cancer Genome Atlas (\acrshort{tcga}) profiled 528 \acrshort{hnscc} clinical and genomic data, which has been standardized and available at a unified data portal, \acrfull{gdc} of the \acrfull{nci}.
%\cite{SeanDavis;MartinMorgan2020}
GDC is available at https://portal.gdc.cancer.gov/projects/TCGA-HNSC.
\acrshort{tcga} offers several computational tools to the public for facilitating cancer research.
Broad \acrfull{gdac} Firebrowse (firebrowse.org, version 1.1.35, 2016\_09\_27) is one of those tools to provide data access to each \acrshort{tcga} disease through a \acrfull{rest} \acrfull{api}.
The 528 \acrshort{tcga} \acrshort{hnscc} patients' clinical information and \acrshort{rnaseq} data were obtained from the Firebrowse \acrshort{rest}ful \acrshort{api} with an R package, FirebrowseR (available at https://github.com/mariodeng/\linebreak
FirebrowseR)\cite{Deng2017}. 
We utilized FirebrowseR with our analysis workflow (see Figure \ref{fig:figure1}, step 1) to receive standardized data frames while avoiding data re-formatting, often causing some errors.

\subsubsection{RNA Sequencing Data} 

The number of protein-coding genes was suggested as 20,500\cite{Clamp2007}. The \acrshort{gdc} Data Portal provided \acrshort{tcga} data has been harmonized and re-aligned \acrlong{rnaseq} data against an official reference genome build (\acrlong{grch38}, \acrshort{grch38}). \acrshort{rnaseq} expression level read counts produced by Illumina HiSeq have been normalized using the \acrfull{fpkm} method, as described in reference\cite{FPKM2017}.
The \acrshort{rnaseq} preprocessor of Broad \acrshort{gdac} picked the \acrfull{rsem} value from Illumina HiSeq/GA2 messenger \acrshort{rnaseq} level\_3 (v2) dataset of \acrshort{nci} \acrshort{gdc}. It made the messenger \acrshort{rnaseq} matrix with log2 transformed for the downstream analysis, as described in their reference\cite{RSEM2016}.
We utilized FirebrowseR's function call, Samples.mRNASeq(cohort = "HNSC", gene=GeneName, format="csv"), to download each \acrshort{rnaseq} data of all \acrshort{hnscc} patients and to save as 20,499 data frame files, named as "HNSCC.mRNA.Exp.[GeneName].\newline
Fire.Rda".
After careful investigation of the genomics dataset, the \acrshort{rnaseq} values of "\acrfull{slca}" and "\acrfull{slcb}" should be considered two distinct expression entities. We concluded that the number of protein-coding genes in the \acrshort{tcga} dataset is 20,500. We removed null expressed genes, which over 50\% of the cohort, to avoid the useless result.

\subsubsection{Clinical Data} 

We utilized FirebrowseR's function call, Samples.Clinical(cohort = "HNSC", format="csv"), to get all 81 clinical features (including pathological data, defined by \acrshort{tcga} \acrshort{gdc} data dictionary: \acrfull{cde}\cite{CDE2019}) of all 528 \acrshort{hnscc} patients, which saved as one data frame file: "HNSCC.clinical.\linebreak
Fire.Rda" (accessed November 2019).

One "HNSCC.clinical.Fire.Rda" tables and 20,500 "HNSCC.mRNA.Exp.\newline
[GeneName].Fire.Rda" tables were transposed and merged by their \newline
\_participant\_barcode (unique patient \acrlong{id}, \acrshort{id}) to yield a data frame with 528 rows (participants) against 20,581 columns (81 clinical features as well as 20,500 protein-coding \acrshort{rnaseq} of cancer specimen).
The clinicopathological features selected for our workflow included gender, age, clinical tumor size, clinical cervical lymph node metastases, clinical distant metastasis assessment, pathological surgical margin, and tobacco exposure with their corresponding survival data.
The tumor size (T), cervical lymph node metastases (N), and distal metastasis status (M) were classified according to \acrfull{ajcc}\cite{Amin2017} along with \acrfull{uicc}\cite{Brierley2016} \acrshort{tnm} system for clinical staging of \acrshort{hnscc}.
% In the Eighth Edition, the AJCC has expanded the use of non-anatomic prognostic factors and biomarkers in assigning prognostic stage groups.
We made data clean by removing duplicated rows and columns.


\subsection{Cutoff Finder Core Engine}

%\label{key:cutoff}
To evaluate the effect of gene expression on the patient's survival, we introduced the stratifying of patients with Kaplan-Meier survival analysis according to each gene's low/high expression.
Our cutofFinder\_func subroutine employs the minimum \textit{P} value approach to recognizing cutoff points in continuous gene expression measurement for patients sub-population.
First, patients were ordered by \acrshort{rnaseq} value (RSEM) of a given gene. Next, patients were stratified at a serial cut (counted by person ranked in 30\% to 70\% percentile of the cohort; please see Figure \ref{fig:figure1} "HNSCC cohort"). The survival risk differences of the two groups were estimated by log-rank test to yield around 165 Kaplan-Meier \textit{P} values for each gene.
Then, the optimal cutoff of \acrshort{rnaseq}, giving the minimum \textit{P} value, was selected by the cutofFinder\_func subroutine.
This iteration method could calculate all possible cutoff of each gene expression in this cohort. At each run of cutofFinder\_func function call for an individual gene, it returned an optimal cutoff value (e.x. 0.027 for gene \acrlong{CAMK2N1}, \acrshort{CAMK2N1}). The optimal cutoff value and its correlated patient grouping size (e.x. low-expression in 262 persons vs. high-expression in 152 persons with gene \acrshort{CAMK2N1}) would be returned to the main program to allow downstream Cox survival analysis. The percentile range we applied as 30\% to 70\% was used to avoid a small grouping effect\cite{Miller1982}\cite{Mizuno2009a}.
In case there was no significant \textit{P} value, a median expression of this gene was set as its cutpoint as usual.

\subsection{Statistical Consideration}

Our workflow has loops to call function survival\_marginSFP(GeneName) with given GeneName to process the survival analysis gene by gene.
We dichotomized the clinicopathological features, which includes gender (male/ female), age at diagnosed (below/beyond 65 year-old), clinical tumor size (T1-2/T3-4), clinical nodal status (negative/positive), clinical distant metastasis (negative/positive), TNM staging (early/late), surgical margin status (negative/positive) and tobacco exposure (low/high). The patients were grouped by an \acrshort{rnaseq} value of each gene, cut at low- or high-expression on an optimal \textit{P} value obtained from the cutofFinder\_func subroutine (see the section of "Cutoff Finder Core Engine"). Pearson's chi-square test was used for these binary variables. Kaplan-Meier survival was analyzed using the log-rank test for two groups OS comparison.
The Cox proportional hazards regression is the widely accepted approach for modeling survival while accounting for confounding factors\cite{Magen2019}. Univariate and multivariate Cox proportional regression model\cite{Andersen1982}, using the "coxph" function in R package "survival", was applied to calculate hazard ratio, \acrfull{ci95} and its significance, and to estimate the independent contributions of each clinicopathological features to the OS.
Results were considered statistically significant when a two-sided \textit{P} $<$ 0.05, or a lower threshold if indicated.
The \acrfull{fdr} ($< 0.05$) could be used to pick up the optimal \textit{P} value to ensure the control for type I error of the Kaplan-Meier survival test during the cutoff finding procedure.
%statistical inference problems in biomedical and genomic data analysis routinely involve the simultaneous test of thousands, or even millions, of null hypotheses. Examples
There were also multiple correlated tests of null hypotheses in the Kaplan-Meier family during our global scanning of protein-coding genes. The stringent Bonferroni correction could result in an adjusted \textit{P} value to ensure the control for type I error. 
%There were multiple correlated tests in the family of Kaplan-Meier survival hypotheses during our global scanning of protein-coding genes. The stringent Bonferroni correction could result in an adjusted P-value to ensure the control for type I error. 

The resulting data, including Kaplan-Meier curves, cumulative \textit{P} value plots, and Cox regression tables, were exported to ".xlsx" and ".Rda" files (by R package "r2excel") for subsequent biomarker selection.


\subsection{Biomarker Selection}

Those genes with prognostic impact, whose hazard ratio $>=1.5$ or $<=0.5$ in both Cox's univariate/multivariate model, were assigned as provisional candidates.
Bonferroni adjusted (Kaplan-Meier) \textit{P} value was used to make a ranking of candidates for the final decision (see Figure \ref{fig:figure1}, step 2).

%%%%%
\section{Conclusions} % 5
% Conclusions: This section is not mandatory, but can be added to the manuscript if the discussion is unusually long or complex.
%The result showed six overexpressed genes (symbol as CAMK2N1, PGK1, SURF4, USP10, NDFIP1, FOXA2) are significantly associated with a poor prognosis of overall survival. Furthermore, the four overexpressed genes (symbol as IL19, FCGBP, IQCN - former symbol as KIAA1683, and NPB) are correlated with better survival.
Our findings suggested 10 candidate biomarkers,CAMK2N1, PGK1, SURF4, USP10, NDFIP1, FOXA2, as well as IL19, FCGBP, IQCN - former symbol as KIAA1683, and NPB, are all heavily associated with the prognosis of \acrshort{os} under optimal cutoff points with stringent Bonferroni \textit{P} values and proper confounders control. They also might be potential common biomarkers for subsequent study.
% welcome for study

%tumor-agnostic research, we plan to explore common biomarkers crossing TCGA diseases
We wish this bioinformatics tool will be available for the broad usage of tumor-agnostic research\cite{Looney2020} to crossing several \acrshort{tcga} diseases to make translational impacts. %Tex: PvalueTex

We aim to promote a reproducible bioinformatics\cite{Preeyanon2014}\cite{Kulkarni2018} workflow allowing successful repetition and extension of analyses based on the TCGA or other in-house HNSCC datasets.
TCGA has not collected holistic features.
%%%%%%%%%%%%%%%%%%%%%%%%%%%%%%%%%%%%%%%%%%
%\section{Patents}

%This section is not mandatory, but may be added if there are patents resulting from the work reported in this manuscript.

%%%%%%%%%%%%%%%%%%%%%%%%%%%%%%%%%%%%%%%%%%
\vspace{6pt} 

%%%%%%%%%%%%%%%%%%%%%%%%%%%%%%%%%%%%%%%%%%
%% optional: The data presented in this study are available in supplementary material.
\supplementary{The following are available online at \linksupplementary{Supplementary_figures.pdf}, Figure S1: The query results from SurvExpress, Figure S2: The query results from HPA, Figure S3: GSE2837 query results from PrognoScan.}
%Table S1: title, Video S1: title.}
%S1
%S2
%S3

% Only for the journal Methods and Protocols:
% If you wish to submit a video article, please do so with any other supplementary material.
% \supplementary{The following are available at \linksupplementary{s1}, Figure S1: title, Table S1: title, Video S1: title. A supporting video article is available at doi: link.} 

%%%%%%%%%%%%%%%%%%%%%%%%%%%%%%%%%%%%%%%%%%
\authorcontributions{
%For research articles with several authors, a short paragraph specifying their individual contributions must be provided. The following statements should be used ``
%Conceptualization, Michael Hsiao; Funding acquisition, Yu-Chuan (Jack) Li; Methodology, Li-Hsing Chi; Resources, Alexander TH Wu; Software, Li-Hsing Chi; Supervision, Yu-Chuan (Jack) Li; Validation, Michael Hsiao; Writing - original draft, Li-Hsing Chi; Writing - review and editing, Alexander TH Wu. 
Conceptualization, Yu-Chuan (Jack) Li; Data curation, Alexander TH Wu; Formal analysis, Li-Hsing Chi; Investigation, Michael Hsiao; Methodology, Michael Hsiao; Resources, Alexander TH Wu; Software, Li-Hsing Chi and Yu-Chuan (Jack) Li; Supervision, Michael Hsiao and Yu-Chuan (Jack) Li; Writing – original draft, Li-Hsing Chi; Writing – review and editing, Alexander TH Wu.
All authors have read and agreed to the published version of the manuscript.}
%'' please turn to the  \href{http://img.mdpi.org/data/contributor-role-instruction.pdf}{CRediT taxonomy} for the term explanation. Authorship must be limited to those who have contributed substantially to the work~reported.


\funding{The APC was funded by Taipei Medical University}

\institutionalreview{Not applicable}

\informedconsent{Not applicable}

\dataavailability{All data process and analyses were performed with R programming language (https://www.r-project.org/, version 4.0.2 2020-06-22) and R packages "firebrowseR", "survival", "reshape", "data.table", "ggplot2", "R.utils", "xlsx", "r2excel", "rJava" and "rms" at Rstudio server (version 1.2.5001) based on Google cloud platform under operation system Linux (Ubuntu LTS, release v18.04.3).
%The supplementary figures are available online.%, tables, and R script codes 
The R script codes and datasets generated during the current study are available at the GitHub repository, https://github.com/texchi2/pvalueTex.} 

\acknowledgments{The results shown here are in whole based upon data generated by the \acrshort{tcga} Research Network: https://www.cancer.gov/tcga.
The tables were generated by latex code with the help from https://github.com/JDMCreator/\newline
LaTeXTableEditor.
We like to thank Dr. Wen-Chang Wang,
%https://orcid.org/0000-0003-3124-2136
College of Medical Science and Technology, Taipei Medical University, for bioinformatics consultation and helps. 
We like to thank Sylvain Hall\'e, Department of Computer Science and Mathematics, Univerit\'e du Qu\'ebec \`a Chicoutimi, Canada, for latex technique help with his TeXtidote.
%We also thank the statistical consultation and helps by Dr. Hsin-Chou Yang of the Institute of Statistical Science, Academia Sinica, Taipei, Taiwan.
We sincerely want to express our thanks to the \acrshort{hnscc} patients who donated their data to \acrshort{tcga} affiliated biobanks.

%\subsection*{Authors' information} % Author details (automatic placed here)

% *** URL Accessed 20 May 2013.
The Mouse Tumor Biology Database. http://tumor.informatics.jax.org/\newline
mtbwi/index.do. Accessed 20 May 2020.}

\conflictsofinterest{The authors declare no conflict of interest.} 

%% Optional
%\sampleavailability{Samples of the compounds ... are available from the authors.}

%%%%%%%%%%%%%%%%%%%%%%%%%%%%%%%%%%%%%%%%%%
%% Only for journal Encyclopedia
%\entrylink{The Link to this entry published on the encyclopedia platform.}

%%%%%%%%%%%%%%%%%%%%%%%%%%%%%%%%%%%%%%%%%%
%% Optional
\abbreviations{The following abbreviations are used in this manuscript:
\noindent 
\renewcommand*{\glsinlinenameformat}[2]{\glstarget{#1} {\textsc{#2}},}
\printglossary[type=\acronymtype, title= , nonumberlist, style=inline]
% Abbreviations

%\begin{tabular}{@{}ll}
%MDPI & Multidisciplinary Digital Publishing Institute\\
%DOAJ & Directory of open access journals\\
%TLA & Three letter acronym\\
%LD & Linear dichroism
%\end{tabular}
}

%%%%%%%%%%%%%%%%%%%%%%%%%%%%%%%%%%%%%%%%%%
%% Optional
%\appendixtitles{no} % Leave argument "no" if all appendix headings stay EMPTY (then no dot is printed after "Appendix A"). If the appendix sections contain a heading then change the argument to "yes".
%\appendix
%\section{}
%\subsection{}
%The appendix is an optional section that can contain details and data supplemental to the main text---for example, explanations of experimental details that would disrupt the flow of the main text but nonetheless remain crucial to understanding and reproducing the research shown; figures of replicates for experiments of which representative data are shown in the main text can be added here if brief, or as Supplementary Data. Mathematical proofs of results not central to the paper can be added as an appendix.

%\section{}
%All appendix sections must be cited in the main text. In the appendices, Figures, Tables, etc. should be labeled, starting with ``A''---e.g., Figure A1, Figure A2, etc. 

%%%%%%%%%%%%%%%%%%%%%%%%%%%%%%%%%%%%%%%%%%
\end{paracol}
\reftitle{References}

% Please provide either the correct journal abbreviation (e.g. according to the “List of Title Word Abbreviations” http://www.issn.org/services/online-services/access-to-the-ltwa/) or the full name of the journal.
% Citations and References in Supplementary files are permitted provided that they also appear in the reference list here. 

%=====================================
% References, variant A: external bibliography
%=====================================
\externalbibliography{yes}
%\bibliography{your_external_BibTeX_file}
%\bibliographystyle{model1-num-names}
\bibliography{TCGA_margin_cutoff.bib, Holistic.bib}


%=====================================
% References, variant B: internal bibliography
%=====================================
%\begin{thebibliography}{999}
% Reference 1
%\bibitem[Author1(year)]{ref-journal}
%Author~1, T. The title of the cited article. {\em Journal Abbreviation} {\bf 2008}, {\em 10}, 142--149.
% Reference 2
%\bibitem[Author2(year)]{ref-book1}
%Author~2, L. The title of the cited contribution. In {\em The Book Title}; Editor1, F., Editor2, A., Eds.; Publishing House: City, Country, 2007; pp. 32--58.
% Reference 8
%\bibitem[Author8(year)]{ref-url}
%Title of Site. Available online: URL (accessed on Day Month Year).
%\end{thebibliography}

% The following MDPI journals use author-date citation: Arts, Econometrics, Economies, Genealogy, Humanities, IJFS, JRFM, Laws, Religions, Risks, Social Sciences. For those journals, please follow the formatting guidelines on http://www.mdpi.com/authors/references
% To cite two works by the same author: \citeauthor{ref-journal-1a} (\citeyear{ref-journal-1a}, \citeyear{ref-journal-1b}). This produces: Whittaker (1967, 1975)
% To cite two works by the same author with specific pages: \citeauthor{ref-journal-3a} (\citeyear{ref-journal-3a}, p. 328; \citeyear{ref-journal-3b}, p.475). This produces: Wong (1999, p. 328; 2000, p. 475)

%%%%%%%%%%%%%%%%%%%%%%%%%%%%%%%%%%%%%%%%%%
%% for journal Sci
%\reviewreports{\\
%Reviewer 1 comments and authors’ response\\
%Reviewer 2 comments and authors’ response\\
%Reviewer 3 comments and authors’ response
%}
%%%%%%%%%%%%%%%%%%%%%%%%%%%%%%%%%%%%%%%%%%
\end{document}

